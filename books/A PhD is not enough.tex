
\chapter{有了博士学位还不够:学术生涯指南}
\Large\textbf{A PhD Is not Enough! A Guide to Survival in Science}
\par \emph{Peter J. Feibelman} \normalsize

\par 作者认为,(自然科学)学术界的生存技能是可以传授的,而非只能通过积累经验获得。

\par \textbf{做研究}:不应只关注研究中的技术性问题及其解决方案,还要清楚研究的背景和意义,在科学体系中的位置。\emph{时至今日,这一点要求已被固化在期刊论文的格式里。} 建议问题导向而非技术导向,关注科学问题的解决。将较大的研究目标分解为一系列小文章,及时发表(论文量子``publon'')。平行开展两个或多个项目。

\par \textbf{选择导师和团队}:相比年轻导师,资深导师“不会与你争抢什么东西”,学生的成功就是他的成功;年轻导师可能会对学生或博士后有所戒备,不愿分享科研思路、研究进展。理想的科研团队应当有清晰的目标,内部相互交流;不仅是导师,每个成员也了解研究的整体方向。

\par \textbf{做博士后}:做出一些成果(谨慎选择长期投入而不能及时出成果的项目);让同事认可自己的能力(多交流)。

\par \textbf{做报告}:自信的气场;提纲需要准备,但不必在开始时介绍;除非在讨论技术进展的场合,不必着重公式等技术细节\emph{(可以列出公式,但只讲方法的思想和应用,如王劲峰分层异质性报告)}。

\par \textbf{职业选择}:高校教授拥有科研上的自由,广受尊重,寒暑假和学术休假,顾问、编写教材等额外收入;任务繁多(授课,参与学院管理,申请基金,审稿、编辑等学术服务),以至没有时间参与具体科研工作。助理教授则承受上述多数不利而只有少数好处(低薪、任务繁多且职业没有保障)。政府和企业研究机构职责相对单一,但需受管理。

\par \textbf{求职面试}:了解听众的需求和对应聘者的期待。表现出积极主动、愿与人交流。有清晰的科研方向、未来两三年的计划;并阐明计划与要加入团队的契合之处。可以向主管领导发邮件确认约定的工作条件。如果必要可以毁约较早的offer,接下之后更好的。

\par \textbf{申请经费}:职业早期不建议申请重大、需要多年积累的项目;或者可以分点时间做,快成功时再申请。