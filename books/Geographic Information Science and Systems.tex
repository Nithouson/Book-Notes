

\chapter{地理信息科学与系统}
\Large\textbf{Geographic Information Science and Systems}
\par \emph{Paul A. Longley, Michael F. Goodchild, David J. Maguire, David W. Rhind} \normalsize

\section{引言}

\par 格言:"Everything that happens, happens somewhere."

\par 涉及地理信息的问题场景分类标准:(1)空间尺度,或涉及到的地理细节的层次。(2)时间尺度。(3)目的,好奇心驱动的科学研究或现实世界中的问题求解。二者使用的方法没有明显区别。

\par Geographic, Geo-spatial, Spatial往往可以混用。后者强调不限于地球表层空间,事实上空间分析也适用于其它星球、宇宙空间、人体空间等。

\par Spatial is Special(GI的特殊性):
\begin{itemize}
    \item 几乎所有的人类活动和决策涉及位置因素,而且后者是十分重要的
    \item 空间数据是高维数据(二/三维位置加属性);数据量可能很大;可能需要多源融合
    \item 在计算机中有不同表达方式;可以在不同尺度进行表达;需要投影到平面
    \item 需要独特的分析和可视化方法
\end{itemize}

\par 知识(Knowledge)分为可编码(codified)和隐性(tacit)知识,前者容易书写和传递,后者难以传递。

\par 地理知识可区分为关于形式的(form,how it looks),和关于过程的(process, how it works)。后者可以用于预测,因而更有价值;一般也具有更强的可泛化性。个案式的地理学(idiographic)更关注形式,强调区域特性;普适式的地理学(nomothetic)寻求普适的过程。两者都是必要的,实际问题的解决需要普适过程知识和具体区域知识的结合。 GIS恰恰实现了这种结合,具体信息存储在数据库中,而普适知识由GIS中的分析方法和工具实现;GIS既提供了不同地点间泛化的途径,又顾及了不同地点间独特的语境。

\par \textbf{地理信息系统}是基于计算机的,用于收集、存储、处理、分析、可视化地理信息的工具。从某种意义上说,整个数字世界已经成为一个巨大的、互联的地理信息系统(Twitter, Flickr, RFID刷卡,信令和行车轨迹);GIS的边界变得模糊,几乎人人都是GIS用户。

\par 地理信息系统的组成部分:硬件, 软件, 数据, (组织管理上的)流程, 人, 网络。硬件包括台式机,笔记本、车载设备等;软件可以是开源或商业软件,在本地或云端;大数据指数据量大、结构复杂,超出传统数据管理和处理工具能力。

\par \textbf{地理信息科学}是使地理信息系统得以实现的一般知识和重要发现,与技术细节相比具有相对的稳定性。其核心动机是世界中时空现象的表达,包括形式和过程两个层面。作为一门科学,其应当具备假设和方法的透明性、客观性、可重复性、结果可验证性、可泛化性。Varenius计划指出GIScience位于个体、计算机、社会三者的交汇点。UCGIS与AAG共同编制了地理信息科学与技术知识体系(GISBoK). 

\par 对于社会和环境科学而言,控制条件实验有时难以实施,带来不确定性;且个体具有自主性,故难以得到严格的规律(law),往往有反例。

\par 场所(Place)是人类对空间的社会建构。空间独立于人而存在,人的交互和经验通过共同的感知和认识将空间塑造为场所。

\par \textbf{发展历史}:开创时期(20世纪六七十年代),商业化时期(20世纪八九十年代),开放和泛用时期(21世纪以来)。1963年加拿大地理信息系统诞生,最初的目的是土地管理中的面积测量;1967年双重独立地图编码(Dual Independent Map Encoding)程序诞生,服务于美国人口普查的街道数字化。70年代末,哈佛大学开发了通用的Odyssey GIS。几乎独立地,60年代末英国制图部门开始尝试计算机制图,到70年代末主要国家制图机构都实现了一定程度上的计算机化。80年代初,计算设备成本下降,GIS产业开始快速发展。1981年ESRI ArcInfo发布。

\par \textbf{学科地位}:
\begin{itemize}
    \item 商业视角:软件产业、数据产业(政府测绘部门、遥感影像公司、OpenStreetMap等)、地理信息服务(提供解决方案)。
    \item 政府视角:商业GIS的最大用户群体;分发地理信息数据,特别是公众事务数据(Public-sector information)。
    \item 信息科学视角:GIScience可以视为信息科学或计算机科学的分支。从技术角度,时空数据库、计算几何的很多议题属于计算机科学范畴,但地理信息的独特性决定了其很多原理与信息科学关系微弱。
    \item 地理学视角:GIS的很多基础来源于地理学中的空间分析传统。GIS的发展与地理学、测绘学、规划、景观等学科紧密相关。Neogeography 用于描述互联网制图和网上地理信息传播的发展,其基础是用户和网站的双向交互,如WikiMapia和OpenStreetMap. 仍有地理学家对地理研究中GIS的使用持怀疑态度。
    \item 社会视角:软件商会利用用户暴露的位置信息,或通过cookie记录偏好,用于用户画像和广告精准投放。志愿者地理信息(Volunteered Geographic Information)指个体自愿提供的地理数据,其内容、范围、收集过程往往定义不完善;志愿者的能力和知识存在差异,数据覆盖也受可达性、个人兴趣等因素影响。有观点批评GIS在社会科学中的应用,认为其代表了逻辑实证论(logical positivism,即只有可以用逻辑解决的问题才有意义)。
\end{itemize}

\par \textbf{社会问题}:
\begin{itemize}
    \item 知识与权力的关系。GIS对地表的表达可能对特定人群有利,少数人群或个体的意见可能被忽视。
    \item 用于军事(对别国的测绘)、侵犯个人隐私(监视)或被恶意目的利用(如开放珍稀动物分布数据引发盗猎,弱势群体数据引发犯罪等)。很多人不知道其个人信息如何被采集、使用。
\end{itemize}

\par 实例:地理家谱学(Geo-Genealogy),分析姓氏的空间分布及其演变过程,也用于研究族群分布。种群基因学(Population Genetics),Walter Bodmer 研究英国人等位基因的空间分布,反映族群入侵、定居的历史进程。

\section{地理数据的性质}

\par 有关空间变异(spatial variation)性质的基本原理:邻近效应,尺度效应,不同地理现象的共变性(covariation,同一位置不同属性存在联系)。
\par \textbf{属性数据的分类}:名义(nominal),有序(ordinal),间隔(interval),比率(ratio)四种类型。循环量(如方位角)需要特殊处理,不能直接平均(359°与1°平均非180°)。若要作为输入特征,一种处理方式是采用方位角的正弦和余弦。
\par 制图中还区分spatially extensive与intensive变量。前者仅对整个区域有意义(如计数),可视为某一密度场在区域内的积分;后者在区域同质情况下对每一点都有意义(如密度,比率),可视为某一场在区域内的平均。一般认为等值区域图(choropleth map)不能用于spatially extensive变量。
\par 空间对象可以是自然的或假想的(如质心),按维度分为点(point)、线(line)、面(area)、体(volume)。实体的表达方式决定了空间变异的分析方法,且与尺度密切相关。
\par Scale一词的多重含义:(a)数据的细节层次(level of details, granularity),即空间分辨率;(b)研究项目的范围(extent);(c)地图的比例尺(representative fraction). \emph{Lam认为有四种含义,还包括空间过程的范围(spatial extent of a spatial process).}

\par \textbf{空间自相关}:由位置和属性的相似性决定。若空间上相似的事物属性也相似,称为正空间自相关,与Tobler第一定律一致;反之为负空间自相关。空间自相关有助于对世界的建模,但不利于地理变量的统计推断和预测(\emph{违反了样本独立性假设})。
\par ``The past is the key to the present''概括了事件发生的时间语境。时间上的因果只有一个方向,解释当下只需要考虑过去;而空间现象的解释需要考虑所有方向。我们认为时间相近的事件关联更强;Tobler地理学第一定律是其在空间上的类比,即相近的位置更相关或相似。Tobler定律意味着空间分布的连续渐变性,现实中也存在不规则突变的反例。
\par 认识空间自相关的两种途径:演绎法(deductive)依据理论推导;归纳法(inductive)从数据出发度量自相关。自相关统计量通常考虑位置相似度(即空间权重$w_{ij}$)与属性相似度$c_{ij}$的交叉乘积$\sum_{i,j}w_{ij}c_{ij}$,可用于空间点或多边形图层的属性,或网络的边权。\emph{假定地理过程不存在异质性,}低空间自相关可能表明存在局部性的影响因子;而高空间自相关往往伴随区域尺度的变异,可能受大尺度因子影响。

\par \textbf{空间异质性}:``the tendency of geographic places and regions to be different from each other''. 空间变异可能是可控(controlled variance)的或不可控的(uncontrolled variance),前者在某一均值附近振荡;后者观测时间越长变化越剧烈。 一般而言,距离越大,空间数据的变幅越大,异质性越强。

\par \textbf{空间采样}:数据采集涉及尺度、采样方式、样本权重。事实上任何地理表达都可以视为采样的结果。采样过程要求采样范围(sample frame)内每个样本被选中的概率已知,由此可借助统计推断将样本性质推广到总体。一般的空间采样方式包括随机采样、系统抽样(等间隔格点)等。系统抽样对于具周期模式的空间分布(如规则街区)可能不具代表性,对此可先划分均匀网格再在格内随机选点。当采样难以覆盖整个区域时,可在有代表性的若干局部区域聚集采样。有时需要划分子区域,进行分层采样,如异质性明显的区域比相对同质的区域需要更小的采样间隔。对于空间结构已知情形,可以设计有针对性的采样方案,如沿样带、等高线采样。

\par \textbf{距离衰减}:体现在某一地点与其它地点的交互或影响力中,如污染物排放、噪声传播、公园游客等;也用于空间插值中采样点的加权。常见的衰减权重包括线性型($w_{ij}=a-bd_{ij}$),幂律型($w_{ij}=d_{ij}^{-b}$),指数型($w_{ij}=e^{-bd_{ij}}$)等,这假定了距离衰减是连续变化、各向同性的,实际有时并非如此(如诊所的服务区受交通状况、自然因素影响,并非圆形缓冲区)。

\par 尺度变换下众多自然、人文现象呈现自相似特性,其带来的不规则空间变异难以用连续光滑函数表达,可用分形维数等概念描述。

\section{地理表达}

\par \textbf{地理表达}: 对地球表层空间的一部分构建模型,尺度覆盖从建筑物到全球。其目的是满足人们对直接感知范围之外的空间和时间的认知需求(如地理大发现时代地图发挥的巨大作用);此外现实世界中的计划也可以利用模型仿真模拟。借助多种形式的表达(个体记忆、照片、文字记录、测量数据等),人们可以集成远超于个体所能的关于世界的知识。本书关注计算机设备中基于0和1的数字模型(digital model),其相比于纸质地图具有易存储、易复制、表达能力强(动态、三维、不必投影到平面)、便于分析的优点。计算机无法表达无限复杂的现实世界,必须限制细节的层次,忽略某些属性或其时态变化。

\par 数据的二进制表示法:整型(short, long)、浮点型(single: 7位有效数字; double: 14位)、ASCII码、BLOB(binary large object,用于图像、音频等)。

\par ``Geographic data link place, time, and attributes.'' 地理数据的最小单元包含位置、时间(可选)和属性信息。``珠峰海拔为8848.86米''可分解为两条原子信息:珠峰的经纬度、这一位置的海拔高度。\emph{这一最小单元即Geo-atom。} 其中地理位置是将地理信息与其它信息区别开来的要素。

\par \textbf{对象模型}:离散的对象模型认为空间本是空的,边界明确的对象占据了空间,对象是更广泛类属的实例。对象模型对于边界不易确定的事物(如山峰、湖泊)面临困难。\emph{认为对象属性均质也是一个局限。} ``Nothing on the natural Earth looks remotely like a table!'' GIS对三维对象的处理能力有限,一般近似为更低维度的对象。交通网络中的overpass和underpass可通过记录一个方向可否转向另一方向实现。

\par \textbf{场模型}:连续的场模型认为地理世界可以用一系列变量描述,这些变量可以在地表每个位置被测量;包括标量场和矢量场。场的变化有是否平滑之分,地块类型可视为一种边界突变的场;人口密度也是一种场,但在面积单元过小(小于个体尺度)时无意义。

\par 对象模型和场模型只是概念模型,仍包含无限的信息量,没有解决数字化表达问题。栅格和矢量是解决这一问题的两种方式。离散对象易于用矢量表达,而连续场的计算机建模有采样点(规则或非规则)、规则网格、不规则多边形、不规则三角网(Triangulated Irregular Network, TIN)、等高线等方式,其中除规则采样点、网格外均属于矢量模型。与对象模型不同,它们并非真实存在的地理对象。

\par \textbf{地理信息综合}(generalization): 对要素的综合体现在特定比例尺下的制图规范中(如决定多大的要素绘在图上)。McMaster和Shea归纳了十种制图综合操作。折线的简化是最常见的制图综合形式之一,可用Douglas-Poiker算法实现。一些GIS已开始支持圆弧、椭圆弧、Bézier曲线等,但其简化和分析方法尚无共识。此外要素的属性信息也涉及综合。

\par 纸质地图在计算机中存储的方式包括数字线划地图(digital line graph)和数字栅格地图(digital raster graphic)。空间数据集来源于纸质地图时,地图的比例尺有时被称为数据集的比例尺。互联网地图可将矢量数据即时渲染为栅格发送到移动客户端,以实现方向调整等需求。

\par 与空间相关的核心概念(如包含、邻近性)和空间数据模型一起构成了本体(ontology),即获取地理知识的框架。

\section{地理参照}

\par 地理参照(georeference; geolocate, geocode与之近义)即在地理信息中指定地理位置的过程,涉及地理位置的表达。

\par \textbf{地理参照系统的要求}:(1)使用范围内的唯一性,每个参照对应唯一地点;(2)含义被该系统使用者共享;(3)时间上的稳定性(反例如地名、行政边界的演变);(4)位置精度(即不确定性,可用面积表示,与分辨率或测量误差有关)。前两项为基本要求。

\par \textbf{地名和POI}: 地名词典(gazetteer)包含官方定义的标准名称,兴趣点(point-of-interest, POI)数据库也已被集成,用于互联网地图服务。由于邮政系统的发展,一个县(county)范围内的地名一般没有重复,但全球范围内地名重复仍很常见。街道名称一般仅在同一城市唯一。同一地物也可能存在多个名称。地名的位置不确定度往往较大;大量地名未被官方确认,只用于当地社群中。Wikimapia的出现一定程度上改变了这一状况,居民可以独立地命名当地地物。事实上在政府测绘部门出现之前的时代,地名命名也是一种个体行为(如制图师Waldseem\"uller对America的命名)。

\par \textbf{邮政地址和编码}: 对邮政业务可达的住宅、办公室提供地理参照。大量自然地物没有邮政地址和编码。若住宅不沿街道一次编号,无法用线性插值估计位置(如日本门牌号按建筑年代编号)。大量GIS应用依赖通信地址到经纬度的转换(\emph{在互联网地图API中被称作地理编码,反向转换为逆地理编码})。

\par \textbf{IP地址}: 每个连接到互联网的设备有唯一的IP地址,可用于概略定位。

\par \textbf{线性参照系统}: 通过从特定点(网络节点)出发,沿特定路径(网络中的边)的距离描述位置,用于公路、铁路、电网、管道网络等。两条路多次相交,或路与自身相交,可能给线性参照带来问题。

\par 基于坐标系的\textbf{度量参照}(metric georeference)理论上可以实现无限高的定位精度,且便于距离计算。

\par 经度$\lambda$的定义可适用于任何旋转体,而纬度$\phi$的定义依赖于椭球体(椭球面垂线与赤道面夹角)。不同椭球体下同一经纬度的位置可相差\SI{100}{\metre}之多。同一经线圈上纬度的\ang{1}约为\SI{111}{\kilo\metre}, 1’约为\SI{1.86}{\kilo\metre}(也即1海里);赤道上经度类似,其它纬线圈经度则向两极减小。以十进制小数表示时,纬度小数点后第5位约为\SI{1}{\metre},受精度限制一般不需要表达更多位数。地表两点大圆距离的计算公式为:
\begin{displaymath}
R\arccos (\sin\phi_1 \sin \phi_2 +\cos \phi_1 \cos \phi_2 \cos(\lambda_1-\lambda_2)).
\end{displaymath}

\par 将地球表面投影到平面的目的包括:(1)平面地图、图像(如航空相片)作为地理信息的输入、输出媒介;(2)栅格本质上是平面的;(3)同时展现地球表面的全貌。

\par 地图投影的性质主要有等角(conformal)、等积(equal area),二者不可得兼。前者保证每一点处不同方向的缩放比例相同,后者保证地图上的面积缩放比例相同。还可分为圆柱(cylindrical),圆锥(conic), 方位或平面(azimuthal,planar)投影;这三个主要类别并未涵盖所有投影。

\par 常见投影:
\begin{itemize}
    \item Plate Carr\'{e}e投影:直接将经纬度对应为x,y;也称unprojected projection,等距正轴圆柱投影,每一点到赤道距离的缩放比例相同。由于经纬度1°的差异可能很大,不适于GIS空间分析。
    \item Lambert等角圆锥投影
    \item Mercator投影:等角正轴切圆柱投影,能保持恒定航向的航线为直线。
    \item UTM(Universal Transverse Mercator):等角横轴割椭圆柱投影,按6°分带后缩放比例在0.9996(中央经线)至1.0004(分带边界)之间。UTM分带坐标系伪东偏移\SI{500}{\kilo\metre},南半球伪北偏移\SI{10000}{\kilo\metre}。分带给跨边界的城市或国家带来困难,高纬地区常改用切点为极点的方位投影。
\end{itemize}

\par 全球导航卫星系统可用于测量经纬度和高程。确定经纬度需要地平线上至少三颗卫星;确定经纬度和高程需要四颗。GPS的平面定位精度约\SI{10}{\metre}(使用差分GPS可提升至\SI{1}{\metre});高程精度约\SI{50}{\metre}。

\par 混搭(mashup)指通过位置关联(类似图层叠加)组合两种或以上互联网服务,实现它们各自不具备的新功能。如将住房列表接口与互联网地图组合,得到住房地图。

\par 地理配准(georegistration)可以根据一组参考点拟合变换方程,将未知坐标系的几何数据或图像配准到已知坐标系的图像。

\section{不确定性}

\par 大多数地理表达不完备,带来不确定性。不确定性(uncertainty)描述数据集内容与其数据表达的现象的差异,可以作为表达质量的笼统描述,不区分与vagueness, fuzziness, imprecision, inaccuracy等概念的差异。不确定性可在概念化(conception)、地理表达(representation)、数据分析(analysis)三个环节中产生并积累。

\par \textbf{概念化中的不确定性}:包括地方(place)和属性(attribute)的概念化。
\par 地方(place)概念化为带边界的分析单元,但很多情况下地理分析没有自然的单元。Barry Smith区分了两种空间边界:自然具有的(\emph{bona fide})和人为划定的(\emph{fiat});地理分析中的单元多数是后者。划分同质区域(homogeneous zone)从而最大化区域间异质性,是区域地理学的基本主题。另一种方式是划分功能区(functional zone)使区内交互量最大、区域间交互量最小;或根据设施的影响范围分划。
\par 属性(类别)标签可能具有主观性、模糊性,模糊逻辑可缓解这一问题,但其隶属度的精确性受争议(仍是主观的)。采用直接或间接的指数代替所描述的现象,也会因二者对应的不完全性产生不确定性。

\par \textbf{表达中的不确定性}:位置表达中,矢量线的简化有较大自由度;出于减少数据量或隐私保护目的,也将点数据(如个体位置)聚合到面单元。栅格影像中覆盖多种类型土地的像元称为混合像元(mixel),当分类数目不变,混合像元比例随分辨率提高而减少。

\par \textbf{分析中的不确定性}:计算结果的误差可利用数据误差的度量和分布分析得到,也可采用重复模拟的方法评估。误差的空间结构(空间自相关性质)对GIS分析操作的不确定性有重要影响,使误差对许多操作的影响减小(如计算两点间距离、坡度),也减少了数据集的``自由度'';但当两数据集叠加时,误差可能会较大。利用模拟分析误差的影响时,应考虑误差的空间自相关。
\par 生态谬误(ecological fallacy)指基于聚合数据对个体特性的不合理推断。分析单元内的异质性会增加生态谬误出现的可能性和严重性。与之相对的是原子谬误(atomistic fallacy),即认为个体孤立于所处的环境。
\par 可变面积单元问题(modifiable areal unit problem):包括尺度效应(scale effect)和分区效应(zoning effect)。双变量相关系数通常随聚合尺度增大而变得更显著(一个实例由Yule和Kendall发现);多变量分析的尺度效应没有一致、可预测的趋势。1984年Openshaw的研究表明,相关和回归分析中的不同预设结果可通过调整分区方式实现;1812年gerrymander事件是一个现实实例。用地学计算(geocomputation)方法进行区域划分,有助于理解输出的敏感性,但难以得到无可争议的最优解。分析单元的划分方式难以用抽样和统计推断的方法处理。作者提倡依据清晰的假设定义分析单元,并用(反映单元内部异质性的)外部数据源验证划分的合理性。

\par \textbf{不确定性的度量方法}:
\par 分类变量:混淆矩阵(confusion matrix),相比于正确率,Kappa指数用随机分类结果进行修正:
\begin{displaymath}
\kappa = \frac{\sum_i c_{ii}-\sum_i c_{.i}c_{i .}/c_{..}}{c_{..}-\sum_i c_{. i}c_{i .}/c_{..}}.
\end{displaymath}
其中$n$为类别数,.代表求和。分类图中的错误不仅存在于类别标签,还存在于区块边界的位置。
\par 连续变量:精度(precision)通常指测量结果有效数字的位数(报告结果时应当与实际准确度相对应,不提供更多位数),也用于描述重复测量时结果的变异程度。准确度(accuracy)描述测量结果(均值)与真实值的差异。两者没有必然联系,高精度的结果仍可能是有偏的(biased)。误差的常用度量指标是均方根误差(RMSE),很多情况下误差符合正态分布。位置的误差可用多元正态分布描述。

\par \textbf{如何应对不确定性?}(1)认识到不确定性是不可避免的,而非装作其不存在;(2)不能将数据当做真实,利用元数据中的数据质量信息评估其可用性;(3)利用多种数据源进行外部校正,如遥感影像和矢量地图、DEM和高程点数据;(4)在报告GIS分析结果时体现对不确定性的认识。

\section{地理信息系统软件}

\par GIS的三个主要组成部分:用户界面(user interface),工具(tools),数据管理器(data manager)。在信息系统的术语中,三者依次对应表示层(presentation),业务逻辑层(business logic),数据服务层(data server)。

\par GIS的三种架构:桌面(desktop),客户-服务器(client-server),云架构(cloud)。GIS的主要类型包括桌面端GIS,互联网制图系统(Web mapping system),服务器GIS,虚拟地球,开发者GIS,移动端GIS等。相比于服务器GIS,互联网制图系统仅提供制图相关功能。开发者GIS是用于GIS应用开发的组件集。

\par GIS的演进:命令行时代,GIS是处理地理数据、产生派生数据的程序(命令)的集合。20世纪80年代末至90年代初,图形用户界面的出现简化了人机交互,高级程序设计语言和API的出现为软件的自定义开发提供了便利,GIS的应用领域大大扩展。近年来互联网服务(web service)成为主流范式,也是云GIS的组成部分。

\par 主要GIS软件:商业软件:Autodesk AutoCAD Map3D, Bentley Map, Esri ArcGIS, Intergraph GeoMedia. 开源软件:Quantum GIS, GRASS, MapServer. 

\section{地理数据建模}

\par ``representation''与``model''意义相近,前者侧重于概念层面,常用于学术探讨,后者偏实践。

\par 数据模型用于在计算机中描述和表达真实世界的一部分,直接决定了可以在数据上进行何种分析操作。数据模型涉及现实世界的三个抽象层次:概念模型、逻辑模型、物理模型。概念模型面向人(human-oriented);逻辑模型面向实现(implementation-oriented),通常用图表表示;物理模型是GIS中的真实实现。场、对象属于概念模型;矢量、栅格属于逻辑模型。

\par \textbf{CAD和计算机制图系统中的数据模型}:简单的点、线、面矢量符号,一般用局部坐标而非真实地理坐标,对象不能附加属性(制图系统可加注记),也无法表达对象间的关系,故不适于地理信息。

\par \textbf{栅格数据模型}:区别于作为对象属性的图片,须有地理参照信息(像元对应地理位置)。常用压缩方法包括游程长度编码(run-length encoding),块码(block encoding,包括四叉树),小波(wavelet)等。前两种是无损压缩;小波法是有损压缩,用于JPEG2000标准。

\par \textbf{矢量数据模型}:分为简单要素和拓扑要素。简单要素也称spaghetti,只存储要素坐标,不存储对象关系。拓扑要素要求``逢交必断'',多边形图层需不重叠地填满平面(planar enforcement),按节点表、弧段表、多边形表组织,弧段表记录左右多边形。拓扑属性有助于验证数据集的完整性,加速基于拓扑关系的查询。网络模型是一种特殊的矢量模型。
\par 不规则三角网(Triangulated Irregular Network,TIN)遵循拓扑数据结构,存储每个三角形的三个顶点和邻接三角形。TIN的优点是三角形密度可以随空间位置变化;缺点是对异常值敏感(无平滑处理),高程的准确表达需要记录关键特征点(如山峰、山谷)。

\par \textbf{对象数据模型}:核心是一组对象(类)及之间的关系。地理对象集合了实体的几何数据、属性数据和方法,定义了实体的状态和行为。统一建模语言(Unified Modeling Language, UML)常用于描述类之间的关系;计算机辅助软件工程(compter-aided software engineering, CASE)工具可基于图表形式的逻辑模型生成物理模型。

\section{数据收集}

\par 数据收集(data collection)分为数据采集(data capture)和数据迁移(data transfer)。前者指数据直接输入系统;后者是导入已存在的数字化数据,最常见的方式是通过某种中间格式。地理数据可分为原始数据(primary data)和派生数据(secondary data),前者专门为GIS项目直接测量得到,如遥感影像、实地测绘(含GNSS、LiDAR)数据;后者为其它目的采集,通常为硬拷贝文件,需要转换到合适的数字格式才能使用,如纸质地形图、相片等。若不计入人力成本,数据收集费用可以占到GIS项目总费用的60\%-85\%。

\par 模拟数据都需要转化为数字化数据才能加入到GIS中,其途径包括扫描、光学字符识别(Optical Character Recognition)、矢量化、立体摄影测量等。\textbf{矢量化}是扫描地图的主要目的之一。手动数字化有点式、流式之分。自动矢量化仍不完善,需要前/后处理,如把虚线边界改为实线、去除盖住边界的注记等。对复杂地图交互(半自动)矢量化更常用。\textbf{摄影测量}与数字化相似,区别是还需要采集高程信息;其产品包括DEM、正射影像、矢量要素及整合得到的三维场景。

\par 数据收集项目需要权衡质量、速度与费用。

\par \vspace{6em}

\section{空间数据库}
\par 相比于文件管理,数据库管理系统(Database Management System, DBMS)的功能优势包括索引、结构化查询语言(Structured/Standard Query Language, SQL)、访问控制、并发处理(事务机制)、备份等。几乎所有大型GIS都使用DBMS技术管理数据。

\par GIS中常用的三类数据库包括关系数据库(Relational DBMS)、对象数据库(Object DBMS)和对象关系数据库(Object-Relational DBMS)。ORDBMS是附加对象处理功能的RDBMS,IBM DB2、Oracle、Microsoft SQL Server都属于此类。

\par 数据库针对空间数据的扩展主要包括空间数据类型、空间对象的运算函数、查询优化器、空间索引等。加入空间扩展的ORDEMS仍不具备完整GIS的分析、制图等功能,往往作为数据与GIS之间的中间工具。SQL的扩展SQL/MM支持了空间数据类型和相关函数。几何对象的运算包括距离、缓冲区、凸包、求交等。

\par Ted Codd提出的若干范式可通过拆分表格实现,以达到减少冗余的目的,查询时需要重新连接。由于大表的连接操作很费时,GIS中图层数据表往往不遵循此类范式。

\par 空间数据库中的\textbf{拓扑}有两种处理方法:标准化模型(normalized model)和物理模型(physical model)。前者按照弧段-节点结构存储,在实体表中记录每个要素与线、面等基本要素的关联;其几何数据无冗余,但查询、编辑操作涉及多表连接,效率较低;Oracle Spatial采用。后者在要素的几何字段中存储完整的几何信息(节点序列),拓扑约束在另外的表中存储,拓扑关系由外部程序在需要时计算;ArcGIS采用。

\par \textbf{索引}一方面可以避免全表扫描,加快查询效率,另一方面其建立、维护费时(特别是更新频繁的数据库),且占据空间大。空间要素索引的三种常见方法是格网、四叉树和R树。\textbf{格网索引}记录每个网格中的要素列表,一般三层级的格网能够满足需求。\textbf{四叉树}(quadtree)是一类将平面递归分为四个象限的索引的总称,包括点四叉树、区域四叉树等。区域四叉树常用于索引线、面、栅格,经过四叉树编码,数据可通过B树索引。\textbf{R树}利用要素的最小外包矩形(Minimum Bounding Rectangle,  MBR)将要素分组,索引性能较好,但更新比格网、四叉树费时。

\par 大多数数据库中的事务是短事务,通过加锁进行并发控制(消极并发策略, pessimistic locking);空间数据库中的编辑、更新可能耗时很长,需要通过版本控制(积极并发策略, optimistic versioning)支持长事务,即为同时产生的更新分配不同的版本,分别记录每个版本的修改,待编辑完成后合并;检测到冲突时自动或交互式地进行处理。

\section{GeoWeb}

\par GeoWeb描述了一种已充分实现的远景:数据、软件、硬件等GIS的组成部分通过网络远程组织;因特网成为一个巨大的GIS. 相比于传统工作模式,数据存储的位置、数据处理的位置不一定要与用户的位置相同;作为关注对象的位置可以与用户位置相同(用户可以在实地使用GIS)。信息基础设施(Cyberinfrastructure)一词用于描述支撑科学活动的高速网络、高性能处理器、分布式传感网和数据集。

\par 不同设备、数据、程序的互操作依赖标准和规范,其中地理信息领域的规范包括OGC (Open Geospatial Consortium) 指制定的简单要素规范、地理标记语言(Geography Markup Language)、互联网服务规范(Web Map/Feature/Coverage Service)。

\par 对象级元数据(Object-Level Metadata, OLM)是一个数据集内容的描述,有助于数据集检索、可用性的评估。最广泛使用的OLM标准是FGDC (US Federal Geographic Data Committee) CSDGM (Content Standards for Digital Geospatial Metadata). 这一标准旨在列出数据集所有已知、有潜在价值的性质,使元数据的生产成本较高,有时采用精简形式的元数据(如Dublin Core标准,不限于空间数据集)。

\par Geolibrary是支持根据地理位置检索的网络数据库;Geoportal是整合众多Geolibrary的搜索平台(如美国官方的geo.data.gov)。

\par Web 2.0用于描述互联网内容更多地由用户创作的现象。

\par 基于位置的服务(Location-based Service)指计算设备可获知自身位置,并根据位置调整所提供信息的服务。定位方法包括GNSS、所连接基站位置、IP地址等。其应用包括紧急求助电话、周边查询与导航等。

\par 两类地理信息分析功能更适合作为服务提供:依赖大数据量、昂贵且更新频繁的数据集(如地理编码、路线查询、地名服务等),依赖专业运行设备的复杂操作。
