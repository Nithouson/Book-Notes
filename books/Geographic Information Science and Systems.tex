
\chapter{地理信息系统与科学}
\Large\textbf{Geographic Information Science and Systems}
\par \emph{Paul A. Longley, Michael F. Goodchild, David J. Maguire, David W. Rhind} \normalsize

\section{引言}

\par 格言:"Everything that happens, happens somewhere."

\par 涉及地理信息的问题场景分类标准:(1)空间尺度,或涉及到的地理细节的层次。(2)时间尺度。(3)目的,好奇心驱动的科学研究或现实世界中的问题求解。二者使用的方法没有明显区别。

\par Geographic, Geo-spatial, Spatial往往可以混用。后者强调不限于地球表层空间,事实上空间分析也适用于其它星球、宇宙空间、人体空间等。

\par Spatial is Special(GI的特殊性):
\begin{itemize}
    \item 几乎所有的人类活动和决策涉及位置因素,而且后者是十分重要的
    \item 空间数据是高维数据(二/三维位置加属性);数据量可能很大;可能需要多源融合
    \item 在计算机中有不同表达方式;可以在不同尺度进行表达;需要投影到平面
    \item 需要独特的分析和可视化方法
\end{itemize}

\par 知识(Knowledge)分为可编码(codified)和隐性(tacit)知识,前者容易书写和传递,后者难以传递。

\par 地理知识可区分为关于形式的(form,how it looks),和关于过程的(process, how it works)。后者可以用于预测,因而更有价值;一般也具有更强的可泛化性。个案式的地理学(idiographic)更关注形式,强调区域特性;普适式的地理学(nomothetic)寻求普适的过程。两者都是必要的,实际问题的解决需要普适过程知识和具体区域知识的结合。 GIS恰恰实现了这种结合,具体信息存储在数据库中,而普适知识由GIS中的分析方法和工具实现;GIS既提供了不同地点间泛化的途径,又顾及了不同地点间独特的语境。

\par \textbf{地理信息系统}是基于计算机的,用于收集、存储、处理、分析、可视化地理信息的工具。从某种意义上说,整个数字世界已经成为一个巨大的、互联的地理信息系统(Twitter, Flickr, 信令和行车轨迹);GIS的边界变得模糊,几乎人人都是GIS用户。

\par 地理信息系统的组成部分:硬件, 软件, 数据, (组织管理上的)流程, 人, 网络。硬件包括台式机,笔记本、车载设备等;软件可以是开源或商业软件,在本地或云端;大数据指数据量大、结构复杂,超出传统数据管理和处理工具能力。

\par \textbf{地理信息科学}是使地理信息系统得以实现的一般知识和重要发现,与技术细节相比具有相对的稳定性。其核心动机是世界中时空现象的表达,包括形式和过程两个层面。作为一门科学,其应当具备假设和方法的透明性、客观性、可重复性、结果可验证性、可泛化性。Varenius计划指出GIScience位于个体、计算机、社会三者的交汇点。UCGIS与AAG共同编制了地理信息科学与技术知识体系(GISBoK). 

\par 对于社会和环境科学而言,控制条件实验有时难以实施,带来不确定性;且个体具有自主性,故难以得到严格的规律(law),往往有反例。

\par 场所(Place)是人类对空间的社会建构。空间独立于人而存在,人的交互和经验通过共同的感知和认识将空间塑造为场所。

\par \textbf{发展历史}:开创时期(20世纪六七十年代),商业化时期(20世纪八九十年代),开放和泛用时期(21世纪以来)。1963年加拿大地理信息系统诞生,最初的目的是土地管理中的面积测量;1967年双重独立地图编码(Dual Independent Map Encoding)程序诞生,服务于美国人口普查的街道数字化。70年代末,哈佛大学开发了通用的Odyssey GIS。几乎独立地,60年代末英国制图部门开始尝试计算机制图,到70年代末主要国家制图机构都实现了一定程度上的计算机化。80年代初,计算设备成本下降,GIS产业开始快速发展。1981年ESRI ArcInfo发布。

\par \textbf{学科地位}:
\begin{itemize}
    \item 商业视角:软件产业、数据产业(政府测绘部门、遥感影像公司、OpenStreetMap等)、地理信息服务(提供解决方案)。
    \item 政府视角:商业GIS的最大用户群体;分发地理信息数据,特别是公众事务数据(Public-sector information)。
    \item 信息科学视角:GIScience可以视为信息科学或计算机科学的分支。从技术角度,时空数据库、计算几何的很多议题属于计算机科学范畴,但地理信息的独特性决定了其很多原理与信息科学关系微弱。
    \item 地理学视角:GIS的很多基础来源于地理学中的空间分析传统。GIS的发展与地理学、测绘学、规划、景观等学科紧密相关。Neogeography 用于描述互联网制图和网上地理信息传播的发展,其基础是用户和网站的双向交互,如WikiMapia和OpenStreetMap. 仍有地理学家对地理研究中GIS的使用持怀疑态度。
    \item 社会视角:软件商会利用用户暴露的位置信息,或通过cookie记录偏好,用于用户画像和广告精准投放。志愿者地理信息(Volunteered Geographic Information)指个体自愿提供的地理数据,其内容、范围、收集过程往往定义不完善;志愿者的能力和知识存在差异,数据覆盖也受可达性、个人兴趣等因素影响。有观点批评GIS在社会科学中的应用,认为其代表了逻辑实证论(logical positivism,即只有可以用逻辑解决的问题才有意义)。
\end{itemize}

\par \textbf{社会问题}:
\begin{itemize}
    \item 知识与权力的关系。GIS对地表的表达可能对特定人群有利,少数人群或个体的意见可能被忽视。
    \item 用于军事(对别国的测绘)、侵犯个人隐私(监视)或被恶意目的利用(如开放珍稀动物分布数据引发盗猎,弱势群体数据引发犯罪等)。很多人不知道其个人信息如何被采集、使用。
\end{itemize}

\par 实例:地理家谱学(Geo-Genealogy),分析姓氏的空间分布及其演变过程。种群基因学(Population Genetics),Walter Bodmer 研究英国人等位基因的空间分布,反映族群入侵、定居的历史进程。

\section{地理数据的性质}

\par 有关空间变异(spatial variation)性质的基本原理:邻近效应,尺度效应,不同地理现象的共变性(covariation,同一位置不同属性存在联系)。
\par \textbf{属性数据的分类}:名义(nominal),有序(ordinal),间隔(interval),比率(ratio)四种类型。循环量(如方位角)需要特殊处理,不能直接平均(359°与1°平均非180°)。若要作为输入特征,一种处理方式是采用方位角的正弦和余弦。
\par 制图中还区分spatially extensive与intensive变量。前者仅对整个区域有意义(如计数),可视为某一密度场在区域内的积分;后者在区域同质情况下对每一点都有意义(如密度,比率),可视为某一场在区域内的平均。一般认为等值区域图(choropleth map)不能用于spatially extensive变量。
\par 空间对象可以是自然的或假想的(如质心),按维度分为点(point)、线(line)、面(area)、体(volume)。实体的表达方式决定了空间变异的分析方法,且与尺度密切相关。
\par Scale一词的多重含义:(a)数据的细节层次(level of details, granularity),即空间分辨率;(b)研究项目的范围(extent);(c)地图的比例尺(representative fraction).

\par \textbf{空间自相关}:由位置和属性的相似性决定。若空间上相似的事物属性也相似,称为正空间自相关,与Tobler第一定律一致;反之为负空间自相关。空间自相关有助于对世界的建模,但不利于地理变量的统计推断和预测(\emph{违反了样本独立性假设})。
\par ``The past is the key to the present''概括了事件发生的时间语境。时间上的因果只有一个方向,解释当下只需要考虑过去;而空间现象的解释需要考虑所有方向。我们认为时间相近的事件关联更强;Tobler地理学第一定律是其在空间上的类比,即相近的位置更相关或相似。Tobler定律意味着空间分布的连续渐变性,现实中也存在不规则突变的反例。
\par 认识空间自相关的两种途径:演绎法(deductive)依据理论推导;归纳法(inductive)从数据出发度量自相关。自相关统计量通常考虑位置相似度(即空间权重$w_{ij}$)与属性相似度$c_{ij}$的交叉乘积$\sum_{i,j}w_{ij}c_{ij}$,可用于空间点或多边形图层的属性,或网络的边权。\emph{假定地理过程不存在异质性,}低空间自相关可能表明存在局部性的影响因子;而高空间自相关往往伴随区域尺度的变异,可能受大尺度因子影响。

\par \textbf{空间异质性}:``the tendency of geographic places and regions to be different from each other''. 空间变异可能是可控(controlled variance)的或不可控的(uncontrolled variance),前者在某一均值附近振荡;后者观测时间越长变化越剧烈。 一般而言,距离越大,空间数据的变幅越大,异质性越强。

\par \textbf{空间采样}:数据采集涉及尺度、采样方式、样本权重。事实上任何地理表达都可以视为采样的结果。采样过程要求采样范围(sample frame)内每个样本被选中的概率已知,由此可借助统计推断将样本性质推广到总体。一般的空间采样方式包括随机采样、系统抽样(等间隔格点)等。系统抽样对于具周期模式的空间分布(如规则街区)可能不具代表性,对此可先划分均匀网格再在格内随机选点。当采样难以覆盖整个区域时,可在有代表性的若干局部区域聚集采样。有时需要划分子区域,进行分层采样,如异质性明显的区域比相对同质的区域需要更小的采样间隔。对于空间结构已知情形,可以设计有针对性的采样方案,如沿样带、等高线采样。

\par \textbf{距离衰减}:体现在某一地点与其它地点的交互或影响力中,如污染物排放、噪声传播、公园游客等;也用于空间插值中采样点的加权。常见的衰减权重包括线性型($w_{ij}=a-bd_{ij}$),幂律型($w_{ij}=d_{ij}^{-b}$),指数型($w_{ij}=e^{-bd_{ij}}$)等,这假定了距离衰减是连续变化、各向同性的,实际有时并非如此(如诊所的服务区受交通状况、自然因素影响,并非圆形缓冲区)。

\par 尺度变换下众多自然、人文现象呈现自相似特性,其带来的不规则空间变异难以用连续光滑函数表达,可用分形维数等概念描述。

\section{地理表达}

\par \textbf{地理表达}: 对地球表层空间的一部分构建模型,尺度覆盖从建筑物到全球。其目的是满足人们对直接感知范围之外的空间和时间的认知需求(如地理大发现时代地图发挥的巨大作用);此外现实世界中的计划也可以利用模型仿真模拟。借助多种形式的表达(个体记忆、照片、文字记录、测量数据等),人们可以集成远超于个体所能的关于世界的知识。本书关注计算机设备中基于0和1的数字模型(digital model),其相比于纸质地图具有易存储、易复制、表达能力强(动态、三维、不必投影到平面)、便于分析的优点。计算机无法表达无限复杂的现实世界,必须限制细节的层次,忽略某些属性或其时态变化。

\par 数据的二进制表示法:整型(short, long)、浮点型(single: 7位有效数字; double: 14位)、ASCII码、BLOB(binary large object,用于图像、音频等)。

\par ``Geographic data link place, time, and attributes.'' 地理数据的最小单元包含位置、时间(可选)和属性信息。``珠峰海拔为8848.86米''可分解为两条原子信息:珠峰的经纬度、这一位置的海拔高度。\emph{这一最小单元即Geo-atom。} 其中地理位置是将地理信息与其它信息区别开来的要素。

\par \textbf{对象模型}:离散的对象模型认为空间本是空的,边界明确的对象占据了空间,对象是更广泛类属的实例。对象模型对于边界不易确定的事物(如山峰、湖泊)面临困难。\emph{认为对象属性均质也是一个局限。} ``Nothing on the natural Earth looks remotely like a table!'' GIS对三维对象的处理能力有限,一般近似为更低维度的对象。交通网络中的overpass和underpass可通过记录一个方向可否转向另一方向实现。

\par \textbf{场模型}:连续的场模型认为地理世界可以用一系列变量描述,这些变量可以在地表每个位置被测量;包括标量场和矢量场。场的变化有是否平滑之分,地块类型可视为一种边界突变的场;人口密度也是一种场,但在面积单元过小(小于个体尺度)时无意义。

\par 对象模型和场模型只是概念模型,仍包含无限的信息量,没有解决数字化表达问题。栅格和矢量是解决这一问题的两种方式。离散对象易于用矢量表达,而连续场的计算机建模有采样点(规则或非规则)、规则网格、不规则多边形、不规则三角网(Triangulated Irregular Network, TIN)、等高线等方式,其中除规则采样点、网格外均属于矢量模型。与对象模型不同,它们并非真实存在的地理对象。

\par \textbf{地理信息综合}(generalization): 对要素的综合体现在特定比例尺下的制图规范中(如决定多大的要素绘在图上)。McMaster和Shea归纳了十种制图综合操作。折线的简化是最常见的制图综合形式之一,可用Douglas-Poiker算法实现。一些GIS已开始支持圆弧、椭圆弧、Bézier曲线等,但其简化和分析方法尚无共识。此外要素的属性信息也涉及综合。

\par 纸质地图在计算机中存储的方式包括数字线划地图(digital line graph)和数字栅格地图(digital raster graphic)。空间数据集来源于纸质地图时,地图的比例尺有时被称为数据集的比例尺。互联网地图可将矢量数据即时渲染为栅格发送到移动客户端,以实现方向调整等需求。

\par 与空间相关的核心概念(如包含、邻近性)和空间数据模型一起构成了本体(ontology),即获取地理知识的框架。

\section{地理参照}

\par 地理参照(georeference; geolocate, geocode与之近义)即在地理信息中指定地理位置的过程,涉及地理位置的表达。

\par \textbf{地理参照系统的要求}:(1)使用范围内的唯一性,每个参照对应唯一地点;(2)含义被该系统使用者共享;(3)时间上的稳定性(反例如地名、行政边界的演变);(4)位置精度(即不确定性,可用面积表示,与分辨率或测量误差有关)。前两项为基本要求。

\par \textbf{地名和POI}: 由于邮政系统的发展,一个县范围内的地名一般没有重复,但全球范围内地名重复仍很常见。街道名称一般仅在同一城市唯一。

\par 基于坐标系的度量参照(metric georeference)理论上可以实现无限高的定位精度,且便于距离的计算。