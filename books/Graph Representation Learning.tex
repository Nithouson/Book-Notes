
\chapter{图表示学习}
\Large\textbf{Graph Representation Learning\footnote{本笔记亦主要参考Jure Leskovec图机器学习讲义(Stanford CS224W,2021)}}
\par \emph{William L. Hamilton} \normalsize

\section{导论}

\par 图是描述和分析有关联(交互)实体的通用语言。图机器学习的核心问题:如何利用关系结构进行更好的预测?

\par 任务分类:
\begin{itemize}
    \item 节点级(Node level):节点分类
    \item 链接级(Link level):链接预测(根据已知边预测其它节点对是否连边,如知识图谱补全,推荐系统中预测用户与物品的连接)
    \item 子图级(Subgraph level):社区探测
    \item 图级(Graph level):图分类(如分子性质预测),图生成,图演化模拟
\end{itemize}

\par 二部图的投影图(folded/projected graph):设两个顶点独立集$U,V$,对$U$的投影图节点为$U$,$u_1,u_2\in U$连边当且仅当存在$v\in V$,$u_1,u_2$在二部图中都与$v$相连。

\par 异质图(heterogeneous graph):区分不同类型节点和边。 $G=(V,E,R,T)$,边定义为$(v_i,r_{ij},v_j)\in E$, $r_{ij}\in R$是边类型,$T(v_i)$是节点类型。在知识图谱中常用。

\par 稀疏图(sparse graph):平均度远小于$N-1$。大多数现实世界的网络都是稀疏图。

\section{传统方法}

\par 传统图机器学习方法需要人为定义节点、节点对、图的特征(表示为向量)用于训练。

\subsection{节点特征}

度、中心性描述节点的重要性。度、聚集系数、图元度向量描述节点所在局部的拓扑属性。

\par \textbf{度}(degree).

\par \textbf{中心性}(centrality): 衡量节点在图中的重要性。
\par 特征向量中心性(eigenvector centrality)满足
\begin{equation}
    c_v=\frac{1}{\lambda}\sum_{u\in N(v)} c_u.
\end{equation}
其中$A$为无向图的邻接矩阵。上式化为$\lambda \mathbf{c}=A\mathbf{c}$,即$\mathbf{c}$为$A$的特征向量。一般取最大特征值对应特征向量\footnote{无向连通图的邻接矩阵不可约。Perron-Frobenius定理指出,不可约非负矩阵最大特征值为正且代数重数为1,从而该特征向量在乘一个常数意义下唯一。}。
\par 介中心性(betweenness centrality):对节点$v$,定义为一对节点所有最短路径中包含$v$的路径所占比例,对所有不含$v$的节点对求和。
\par 接近中心性(closeness centrality):对给定节点,定义为该节点到其余节点最短路径长度之和的倒数。

\par \textbf{聚集系数}(clustering coefficient):对节点$v$,考虑其邻接节点(设有$k_v$个)的导出子图,其边数与$\binom{k_v}{2}$之比。反映邻域连通程度。
\par \textbf{图元度向量}(graphlet degree vector):图元即较小的连通有根图。令根节点与给定节点对应,对每个图元统计与之同构的导出子图的数量\footnote{如考虑节点数2-5的图元,得到73维向量。}。事实上度统计$K_2$的数量,聚集系数统计$K_3$的数量。


\subsection{节点对特征}

\par \textbf{距离}:网络最短距离。
\par \textbf{局部邻域重叠}:公共邻居数;
\par Jaccard指数:
\begin{equation}
    \frac{\vert N(v_1)\cap N(v_2)\vert}{\vert N(v_1)\cup N(v_2)\vert}
\end{equation}
Adamic-Adar指数:
\begin{equation}
    \sum_{u\in N(v_1)\cap N(v_2)} \frac{1}{\log k_u}.
\end{equation}
局限性是如果没有公共邻居,局部邻域重叠总是0.

\par \textbf{Katz指数}:对简单无向图邻接矩阵乘方,可得到一对节点间给定长度的路径数。设$\beta \in (0,1)$为衰减因子,节点$v_i,v_j$的Katz指数定义为
\begin{equation}
    S_{ij}=\sum_{l=1}^\infty \beta^l A^l_{ij}.
\end{equation}
矩阵形式为
\begin{equation}
    S=\sum_{l=1}^\infty \beta^l A^l=(I-\beta A)^{-1}-I.
\end{equation}

\subsection{图特征}
\par \textbf{图核}(graph kernel)给出两个图的相似度: $K(G,G')=\Phi(G)^T \Phi(G')\in \mathbb{R}$.
\par 图元核(graphlet kernel):

\par Weisfeiler-Lehman核:

\section{图嵌入}

\par 将图的每个节点映射到一个$n$维向量,节点越相似,向量的距离越近。

\section{图神经网络}

\section{图生成模型}

\section{其它主题}
