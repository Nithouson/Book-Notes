\chapter{随椋鸟飞行:复杂系统的奇境}
\Large\textbf{In un volo di storni: Le meraviglie dei sistemi complessi}
\par \emph{Giorgio Parisi} \normalsize

\par \textbf{与椋鸟齐飞}:遵循简单规律的个体组成的群体可能具有复杂的集体行为。椋鸟群能快速变化阵型,既不相撞,又不散开。通过对鸟群轨迹的三维重建,作者团队发现鸟阵边缘的密度比中心更高,可能是为了避免落单的鸟被游隼袭击;鸟与前后同伴的距离远于左右同伴。更重要的是,鸟群中的相互作用取决于最邻近个体间的联系,而非距离远近;鸟群转弯时每只鸟正是依靠邻近鸟的位置进行自我调节。

“在物理学和数学领域……科学研究就像诗歌创作一样,没有任何迹象表明创作过程的艰辛,以及与之相伴的怀疑与彷徨。”
