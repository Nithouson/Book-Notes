\chapter{随椋鸟飞行:复杂系统的奇境}
\Large\textbf{In un volo di storni: Le meraviglie dei sistemi complessi}
\par \emph{Giorgio Parisi} \normalsize

\par \textbf{与椋鸟齐飞}:遵循简单规律的个体组成的群体可能具有复杂的集体行为。椋鸟群能快速变化阵型,既不相撞,又不散开。通过对鸟群轨迹的三维重建,作者团队发现鸟阵边缘的密度比中心更高,可能是为了避免落单的鸟被游隼袭击;鸟与前后同伴的距离远于左右同伴。更重要的是,鸟群中的相互作用取决于最邻近个体间的联系,而非距离远近;鸟群转弯时每只鸟正是依靠邻近鸟的位置进行自我调节。

“在物理学和数学领域……科学研究就像诗歌创作一样,没有任何迹象表明创作过程的艰辛,以及与之相伴的怀疑与彷徨。”

\par \textbf{相变与自旋玻璃}:相变是微观粒子的一种集体行为。一级相变在系统接近临界点时没有预兆,且需要潜热,如冰融化成水。二级相变在接近临界温度时连续发生,如磁铁受热失去磁性。自旋玻璃是一种金属合金,高温下表现与通常磁性系统一致,低于临界温度则变化缓慢,行为类似于玻璃。自旋玻璃模型可视为Ising模型的扩展,后者相邻自旋趋向一致,前者一部分相邻自旋趋向相反。序参量指系统状态转换中的特征参数(如铁磁相变中的磁化强度)。作者发现需要用函数(无穷多个数)作为序参量,其物理意义是系统可能同时处于无数个相。
