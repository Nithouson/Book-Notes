\chapter{计量经济学导论:现代观点}
\Large\textbf{Introductory Econometrics: A Modern Approach}
\par \emph{Jeffrey M. Wooldridge} \normalsize

\section{导论}

\par 计量经济学是数理统计的一个分支,主要研究\textbf{非实验经济数据}收集与分析的固有问题。非实验数据(non-experimental data)也称观测数据(observational data),区别于(自然科学中)控制实验得到的数据,强调研究者只是被动的数据收集者。

\par 计量模型中自变量的识别依据经济理论或直觉。\textbf{误差项}(error term)或干扰项(disturbance term)包含了不可观测及未识别的变量,以及度量自变量时的误差。

\par 数据集分类:(1)\textbf{横截面数据}(cross-sectional data), 同一时间段(或不考虑时间差异)多个样本,通常假定是随机抽样得到(即样本独立同分布)。(2)\textbf{时间序列数据}(time series data),一个或多个变量不同时间的观测值,一般不能假定观测独立于时间,需考虑自相关、季节效应等。(3)\textbf{混合横截面数据}(pooled cross-section data),多个时间段横截面数据的组合(每个时间段对应不同随机样本)。(4)\textbf{面板数据}(panel data),多个样本的时间序列,区别于混合横截面数据,每个单元都被重复观测;可以控制观测单元无法观测的特征,有助于因果推断和时间滞后的研究。

\par 研究两个变量的关系时往往需要其它(相关)因素不变(ceteris paribus);退一步的要求是所关注自变量的选择独立于其它相关因素,从而可以视为实验数据。非实验数据的特征导致难以达到识别因果效应(casual effect)的目标。

\section{线性回归模型}

\subsection{模型假定}
\par \textbf{(1)线性于参数}:总体模型为
\begin{equation}
    y=\beta_0+\beta_1 x_1+\dots+\beta_k x_k +u
\end{equation}
其中$\beta_0$称截距(intercept),$\beta_j(j=1,\dots,k)$称斜率参数(slope parameter),$u$为误差项。
\par 这一模型称为多元线性回归(multiple linear regression)模型;当$k=1$时,称为简单线性回归(simple linear regression)模型。

\par \textbf{(2)随机抽样}:设有来自上述总体模型的含$n$个观测的随机样本$\{(x_{i1}, \dots, x_{ik}, y_i)\}_{i=1}^N$,也即每个样本满足
\begin{equation}
    y_i=\beta_0+\beta_1 x_{i1}+\dots+\beta_k x_{ik} +u_i.
\end{equation}

\par \textbf{(3)不存在完全共线性}:样本中没有一个自变量是常数,自变量之间也没有严格线性关系。
\par 这一假定保证了普通最小二乘法(ordinary least squares, OLS)估计的存在唯一性。自变量之间的非线性依赖关系是被允许的。

\par \textbf{(4)零条件均值}:给定自变量任何值,误差期望为0.
\begin{equation}
    E(u\vert x_1, \dots, x_k)=0.
\end{equation}
\par 这一假定保证了OLS参数估计的无偏性。自变量与误差独立是一个更强的条件。

\subsection{最小二乘法}

\par 给定自变量、因变量的一个样本,被估计的方程,即样本回归函数(sample regression function)形如
\begin{equation}
    \hat{y}=\hat{\beta}_0+\hat{\beta}_1 x_1+\dots+\hat{\beta}_k x_k
\end{equation}
观测$i$的拟合值为
\begin{equation}
    \hat{y}_i=\hat{\beta}_0+\hat{\beta}_1 x_{i1}+\dots+\hat{\beta}_k x_{ik}
\end{equation}
残差(residual)定义为
\begin{equation}
    \hat{u}_i = y_i-\hat{y}_i.
\end{equation}
普通最小二乘法以残差平方和最小为依据,