\chapter{计量经济学导论:现代观点}
\Large\textbf{Introductory Econometrics: A Modern Approach}
\par \emph{Jeffrey M. Wooldridge} \normalsize

\section{导论}

\par 计量经济学是数理统计的一个分支,主要研究\textbf{非实验经济数据}收集与分析的固有问题。非实验数据(non-experimental data)也称观测数据(observational data),区别于(自然科学中)控制实验得到的数据,强调研究者只是被动的数据收集者。

\par 计量模型中自变量的识别依据经济理论或直觉。\textbf{误差项}(error term)或干扰项(disturbance term)包含了不可观测及未识别的变量,以及度量自变量时的误差。

\par 数据集分类:(1)\textbf{横截面数据}(cross-sectional data), 同一时间段(或不考虑时间差异)多个样本,通常假定是随机抽样得到(即样本独立同分布)。(2)\textbf{时间序列数据}(time series data),一个或多个变量不同时间的观测值,一般不能假定观测独立于时间,需考虑自相关、季节效应等。(3)\textbf{混合横截面数据}(pooled cross-section data),多个时间段横截面数据的组合(每个时间段对应不同随机样本)。(4)\textbf{面板数据}(panel data),多个样本的时间序列,区别于混合横截面数据,每个单元都被重复观测;可以控制观测单元无法观测的特征,有助于因果推断和时间滞后的研究。

\par 研究两个变量的关系时往往需要其它(相关)因素不变(ceteris paribus);退一步的要求是所关注自变量的选择独立于其它相关因素,从而可以视为实验数据。非实验数据的特征导致难以达到识别因果效应(casual effect)的目标。