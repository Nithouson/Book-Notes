\chapter{计量经济学导论:现代观点}
\Large\textbf{Introductory Econometrics: A Modern Approach}
\par \emph{Jeffrey M. Wooldridge} \normalsize

\section{导论}

\par 计量经济学是数理统计的一个分支,主要研究\textbf{非实验经济数据}收集与分析的固有问题。非实验数据(non-experimental data)也称观测数据(observational data),区别于(自然科学中)控制实验得到的数据,强调研究者只是被动的数据收集者。

\par 计量模型中自变量的识别依据经济理论或直觉。\textbf{误差项}(error term)或干扰项(disturbance term)包含了不可观测及未识别的变量,以及度量自变量时的误差。

\par 数据集分类:(1)\textbf{横截面数据}(cross-sectional data), 同一时间段(或不考虑时间差异)多个样本,通常假定是随机抽样得到(即样本独立同分布)。(2)\textbf{时间序列数据}(time series data),一个或多个变量不同时间的观测值,一般不能假定观测独立于时间,需考虑自相关、季节效应等。(3)\textbf{混合横截面数据}(pooled cross-section data),多个时间段横截面数据的组合(每个时间段对应不同随机样本)。(4)\textbf{面板数据}(panel data),多个样本的时间序列,区别于混合横截面数据,每个单元都被重复观测;可以控制观测单元无法观测的特征,有助于因果推断和时间滞后的研究。

\par 研究两个变量的关系时往往需要其它(相关)因素不变(ceteris paribus);退一步的要求是所关注自变量的选择独立于其它相关因素,从而可以视为实验数据。非实验数据的特征导致难以达到识别因果效应(casual effect)的目标。

\section{线性回归模型}

\subsection{模型假定}
\par \textbf{(1)线性于参数}:总体模型为
\begin{equation}
    y=\beta_0+\beta_1 x_1+\dots+\beta_k x_k +u
\end{equation}
其中$\beta_0$称截距(intercept),$\beta_j(j=1,\dots,k)$称斜率参数(slope parameter),$u$为误差项。
\par 这一模型称为多元线性回归(multiple linear regression)模型;当$k=1$时,称为简单线性回归(simple linear regression)模型。

\par \textbf{(2)随机抽样}:设有来自上述总体模型的含$n$个观测的随机样本$\{(x_{i1}, \dots, x_{ik}, y_i)\}_{i=1}^N$,也即每个样本满足
\begin{equation}
    y_i=\beta_0+\beta_1 x_{i1}+\dots+\beta_k x_{ik} +u_i.
\end{equation}

\par \textbf{(3)不存在完全共线性}:样本中没有一个自变量是常数,自变量之间也没有严格线性关系。
\par 这一假定保证了普通最小二乘法(ordinary least squares, OLS)估计的存在唯一性。假定的成立必要条件是样本容量$n\ge k+1$。自变量之间的非线性依赖关系是被允许的。

\par \textbf{(4)零条件均值}:给定自变量任何值,误差期望为0.
\begin{equation}
    E(u\vert x_1, \dots, x_k)=0.
\end{equation}
\par 这一假定保证了OLS参数估计的无偏性。其蕴含$\text{Cov}(u,x_j)=0, j=1,\dots,k$;而自变量与误差独立是一个更强的条件。违背此假定的情形包括因变量与自变量关系的误设(如应为二次函数但模型未包括二次项)、遗漏变量(且该变量与任一自变量相关)等。
\par 假定(1)(4)可写成
\begin{equation}
    E(y\vert x_1,\dots,x_k)=\beta_0+\beta_1 x_1+\dots+\beta_k x_k
\end{equation}
这称为总体回归函数(population regression function)。

\par \textbf{(5)同方差性(homoskedasticity)}:给定自变量任何值,误差方差相同,
\begin{equation}
    \text{Var}(u\vert x_1, \dots, x_k)=\sigma^2
\end{equation}
也即$\text{Var}(y\vert x_1, \dots, x_k)=\sigma^2$。误差方差随自变量变化,称为异方差性(heteroskedasticity)。
\par 假设(1)-(5)称为横截面数据回归的Gauss-Markov假定。

\subsection{最小二乘法}

\par 给定自变量、因变量的一个样本,被估计的方程,即样本回归函数(sample regression function)形如
\begin{equation}
    \hat{y}=\hat{\beta}_0+\hat{\beta}_1 x_1+\dots+\hat{\beta}_k x_k
\end{equation}
其中$\hat{\beta}_0$称截距估计值(intercept estimate),$\hat{\beta}_j(j=1,\dots,k)$称斜率参数(slope estimate)。观测$i$的拟合值为
\begin{equation}
    \hat{y}_i=\hat{\beta}_0+\hat{\beta}_1 x_{i1}+\dots+\hat{\beta}_k x_{ik}
\end{equation}
残差(residual)定义为
\begin{equation}
    \hat{u}_i = y_i-\hat{y}_i.
\end{equation}
区别于不可观测的误差,残差是可以计算得到的。普通最小二乘法最小化残差平方和,得到如下一阶条件(first order conditions):
\begin{align}
    &\sum_{i=1}^N \hat{u}_i=0\\
    &\sum_{i=1}^N \hat{u}_ix_{ij}=0, j=1,\dots,k
\end{align}
这也可以由零条件均值假设推出:上式为$Eu=0, E(x_ju)=0$在样本中的对应。

\par 斜率估计值的一个公式为
\begin{equation}
    \hat{\beta}_j=\frac{\sum_{i=1}^n \hat{r}_{ij}y_i}{\sum_{i=1}^n \hat{r}_{ij}^2}
\end{equation}
其中$\hat{r}_{ij}$为$x_j$对$x_1,\dots,x_{j-1},x_{j+1},\dots,x_k$的回归残差。右端即$y$对该残差进行简单线性回归的斜率估计,表明多元回归度量了“排除其他变量影响”时的关系。

\par 定义总平方和(total sum of squares, SST), 解释平方和(explained sum of squares, SSE), 残差平方和(sum of squared residuals, SSR)如下:
\begin{align}
    \text{SST}&=\sum_{i=1}^n (y_i-\bar{y}^2)\\
    \text{SSE}&=\sum_{i=1}^n (\hat{y}_i-\bar{y}^2)\\
    \text{SSR}&=\sum_{i=1}^n u_i^2
\end{align}
可以证明SST=SSR+SSE。定义决定系数(coefficient of determination,R-squared,$R^2$):
\begin{equation}
    R^2=1-\text{SSR}/\text{SST}\label{eq:Rsquared}
\end{equation}
$R^2$同时是实际值$y_i$和拟合值$\hat{y}_i$相关系数的平方。注意这只对于线性回归成立,在其它模型下,二者可能不同;按式(\ref{eq:Rsquared})计算,$R^2$可能为负;但按相关系数平方计算,$R^2$总在[0,1]中。

\subsection{OLS估计量的期望和方差}

\par 在假定(1)-(4)下,OLS估计量是总体参数的无偏估计量,即
\begin{equation}
    E(\hat{\beta}_j)=\beta_j, j=0,1,\dots,k
\end{equation}
\par 在假定(1)-(5)下,OLS斜率估计量以自变量为条件的方差为
\begin{equation}
    \text{Var}(\hat{\beta}_j)=\frac{\sigma^2}{\text{SST}_j (1-R_j^2)}, j=0,1,\dots,k \label{eq:Variance}
\end{equation}
其中$SST_j=\sum_{i=1}^n (x_{ij}-\bar{x}_j)^2$,$R_j^2$是$x_j$对所有其它自变量回归得到的$R^2$。
\par 在假定(1)-(5)下,
\begin{equation}
    \hat{\sigma}^2=\text{SSR}/(n-k-1)
\end{equation}
是误差方差的无偏估计。其平方根$\hat{\sigma}$称回归标准差(standard error of regression, SER)。由此得到OLS斜率估计量标准差的估计:
\begin{equation}
    \text{se}(\hat{\beta}_j)=\hat{\sigma}/\sqrt{\text{SST}_j(1-R_j^2)}
\end{equation}
随着样本量的增大,其以$1/\sqrt{n}$速率收敛到0.

\subsection{Gauss-Markov定理}

\par 在假定(1)-(5)下,对任何$j=0,1,\dots,k$,设$\Tilde{\beta}_j$是$\beta_j$的任一无偏估计量,且具有形式$\Tilde{\beta}_j=\sum_{i=1}^n w_{ij}y_i$,其中$w_{ij}$是自变量值的函数,则有
\begin{equation}
    \text{Var}(\Tilde{\beta}_j)\ge \text{Var}(\hat{\beta}_j)
\end{equation}
也即OLS估计量是最优(方差最小)的线性无偏估计量(best linear unbiased estimate)。

\subsection{遗漏变量偏误}

\par 若包含了无关变量,即总体模型中事实上某个$\beta_j=0$,其它自变量斜率估计值的无偏性不受影响。

\par 若遗漏了有关变量,则所有OLS估计量都可能存在偏误,称遗漏变量偏误(omitted variable bias)。不失一般性,设遗漏变量$x_k$后的OLS估计量为$\Tilde{\beta}_j,j=0,1,\dots,k-1$,$x_k$对$x_1,\dots,x_{k-1}$回归的斜率估计值为$\Tilde{\delta}_j,j=1,\dots,k-1$,则有
\begin{equation}
    \Tilde{\beta}_j=\hat{\beta}_j+\hat{\beta}_k\Tilde{\delta}_j
\end{equation}
其关于自变量的条件期望为
\begin{equation}
    E(\Tilde{\beta}_j)=\beta_j+\beta_k\Tilde{\delta}_j
\end{equation}
故除非$x_k$是无关变量,或回归系数$\Tilde{\delta}_j=0$。有时可以利用$\beta_k$和$\Tilde{\delta}_j$的符号推测遗漏变量偏误的方向。
\par 对于OLS估计量的方差而言,遗漏变量会使其减小(\ref{eq:Variance})。但方差会随样本量的增大而减小,而期望的偏差不会,故大样本情况下更倾向于使用完整模型,而不是为避免多重共线性加剧遗漏变量。

\subsection{多重共线性}

\par 多重共线性(multicollinearity)指自变量之间高度相关(但不完全相关)。多重共线性不影响OLS估计量的无偏性,但会使其方差增大。
\par 方差膨胀因子(variance inflation factor, VIF)定义为:
\begin{equation}
    \text{VIF}=1/(1-R_j^2)
\end{equation}
不应依据$R^2_j$或VIF值简单推断多重共线性的影响,因为OLS估计量的标准差还取决于误差方差和自变量的样本波动。此外若关注的变量与出现多重共线性的变量不相关,多重共线性不影响关注变量的方差。

\subsection{过原点回归}
\par 若利用最小二乘法估计不含截距项的模型,则残差的样本均值不一定为0;若$\beta_0\neq 0$,OLS的斜率估计量将有偏。若$\beta_0=0$,带截距回归的OLS斜率估计量方差比过原点回归大。

\subsection{非线性关系建模}
\par 常半弹性模型:$\ln y=\beta_0+\beta_1 x$,解读为$x$每增加1,$y$增加100$\beta_1$\%.
\par 常弹性模型:$\ln y=\beta_0+\beta_1 \ln x$,解读为$x$每增加1\%,$y$增加$\beta_1$\%.