
\chapter{计量地理学}
\Large\textbf{Quantitative Geography: Perspectives on Spatial Data Analysis}
\par \emph{A. Stewart Fotheringham, Chris Brunsdon, Martin Charlton} \normalsize

\section{概论}
\par 计量地理研究20世纪80年代早期到90年代中期经历低迷,主要在人文地理学领域(自然地理学受影响不大)。其原因包括支撑计量地理学的实证主义哲学\footnote{实证主义(positivism)认为所有的真知都可通过观察、实验、逻辑推理等科学方法证实。}的衰落,马克思主义、后现代主义、结构主义、人文主义等新思潮的兴起(地理学具有批判现有范式、追求新范式的特点)。有观点认为计量方法的批评者大多对其了解不够,认为其相对困难。

\par 对计量地理学的批评包括过于注重普适规律,对个体建模不关注认知与行为过程等。作者指出计量地理学的新发展(如局部分析、融入认知和心理过程的空间交互建模)使上述批评不再适用。事实上,自然主义的观点(“地理即物理”,追求普适定律和全局关系),更多被自然地理而非人文地理学者采用。

\par 计量方法提供了认识空间过程的有效、可靠手段。即使对于难以完全量化的人类行为,计量方法也能量化其中可度量的影响因素(如距离阻碍)。

\par 对空间数据特殊性质(\emph{空间效应})的认识是计量地理学发展成熟的标志,计量地理学者由其它学科技术方法的使用者转变为空间数据分析思想的输出者。

\par 写作本书之时计量地理学的进展体现在局部分析方法、探索性数据可视化、地学计算(Geocomputation)等。地理计算强调空间数据定量分析中对计算机计算能力的运用,如Moran's I的显著性检验中,使用t统计量不视为地理计算,Monte Carlo模拟则视为地理计算,后者规避了理论分布可能不满足的问题。

\section{数据探索性可视化}

\subsection{单变量方法}

\par \textbf{茎叶图}.
\par \textbf{箱线图}(boxplot): 五数概括包括最小值$x_{\min}$、1/4分位数$Q_1$、中位数$Q_2$、3/4分位数$Q_3$、最大值$x_{\max}$。箱线图的矩形边界表示上下四分位数,矩形中的横线表示中位数;竖线可延伸到最大、最小值,或延伸到\footnote{也有时延伸到10\%分位数和90\%分位数}
\begin{equation}
    \min \{x_{\max}, Q_2+1.5(Q_3-Q_1)\}
\end{equation}
及
\begin{equation}
    \max \{x_{\min}, Q_2-1.5(Q_3-Q_1)\}.
\end{equation}
超出此范围的异常值用点标出。

\par\textbf{直方图和频数折线图}:可视为概率密度函数的近似,直方图假设每个分组内概率密度函数为定值,折线图假设为分段线性函数。Terrell和Scott对二者分别给出极大平滑的保守分组个数$(2n)^{1/3}$和$(73.5n)^{1/5}$($n$为观测数),可能存在过度平滑,但更接近真实模式。

\par \textbf{核密度估计}:核$g(\cdot)$是均值为0、方差为1的概率密度函数。对于标量数据,以每个$x_i$为中心生成核,再对所有核求平均得到概率密度函数:
\begin{equation}
    \hat{f}(x)=\frac{1}{n}\sum_{i=1}^n \frac{1}{h}g\left(\frac{x-x_i}{h}\right).
\end{equation}
$h$为带宽,过大则掩盖细节,过小则导致尖峰估计。Terrell给出极大平滑规则:
\begin{equation}
    h \approx s\left(\frac{243\int g^2(x)\text{d} x}{35n}\right)^{1/5}.
\end{equation}
$s$是样本标准差。

\par \textbf{地图}:不同可视化窗口间的关联,特别是数据图与地图的关联很有帮助。

\subsection{多变量方法}

\par 探索的多变量数据特征包括:聚类、异常、趋势。

\par \textbf{散点图矩阵}:$m$个变量的矩阵为$m$行$m$列(对角线为空),同一行具有相同的$y$轴变量,同一列具有相同的$x$轴变量。局限性是只能表示两个变量间的关系。

\par \textbf{平行坐标图}:数据$(x_1,\dots,x_m)$表示为折线$(1,x_1)(2,x_2)\dots(m,x_m)$。实际应用需对各特征标准化处理。可同时表示$m$个变量的信息,但图形模式依赖坐标轴排列顺序(考虑左右对称,$m$个变量有$m!/2$种情形)。

\par \textbf{径向坐标可视化}(Radial Coordinate Visualization, RADVIZ):设$m$个点均匀排布在单位圆周上,位置向量分别为$S_1,\dots,S_m$,数据$(x_1,\dots,x_m)$表示为平面上的点,位置向量为
\begin{equation}
    \mathbf{u}=\frac{\sum_{j=1}^m x_j S_j}{\sum_{j=1}^m x_j}
\end{equation}
其物理解释为:设小球分别通过弹性系数为$x_j$的弹簧与$S_j$相连,则平衡位置为$\mathbf{u}$。

\par 当每个$x_j$非负时,$\mathbf{u}$在$S_1,\dots,S_m$的凸包内。通常对每个变量进行0-1标准化。这一映射当然不是单射,只要每个$x_j$相等,$\mathbf{u}$都取原点。RADVIZ对成分数据(向量各分量之和为1)更适用。考虑旋转和镜像对称,$m$个变量有$(m-1)!/2$种不同的映射;实践中可根据某种目标(如$\mathbf{u}$的方差)做离散搜索优化,也可通过交互控制选取。

\par \textbf{投影寻踪}(Projection Pursuit):指根据要探测的特征选取某个高维特征空间到二(三)维的投影,属于连续优化问题。探测的目标可以是偏离高斯分布的情况(中心孔洞或偏态分布),或最小化平均最邻近距离(Mean Nearest-Neighbor Distance)来探测聚类特征。与之相对的是总体巡查(Grand Tour),投影方向动态变化,以显示所有可能的投影。引导性总体巡查播放投影寻踪中的迭代优化过程。

\section{点模式分析}
\par 点模式分析主要检测点集空间分布的两种现象:聚集和分散,即聚集程度大于或小于随机分布;比较同一区域多组点集分布间的关系,或点集与参考场间的关系(如病例相对于人口密度的集中程度)。

\par \textbf{探索性分析}:绘制空间散点图(注意检查是否有重合点,如有可改用气泡图);计算点集重心,标准距离(到重心的均方根距离)\footnote{当点集不止一个聚簇时,重心和标准距离可能意义不大。};
\begin{equation}
    d_s = \sqrt{\frac{1}{n}\sum_{i=1}^n (x_i-\bar{x})^2+(y_i-\bar{y})^2}
\end{equation}
计算最邻近距离(Nearest-Neighbor Distance)的分布。

\par \textbf{点过程强度}:设$X$是研究区$R$内的点集,$x$点处的强度定义为
\begin{equation*}
    \lambda(x)=\lim\limits_{r\to 0}\frac{E(\vert X\cap U(x,r) \vert)}{\pi r^2}
\end{equation*}
其与概率密度差一个常数:
\begin{equation}
    f(x)=\frac{\lambda(x)}{E(\vert X\cap R \vert)}
\end{equation}
考虑不同位置的相互作用,定义二阶强度为
\begin{equation*}
\gamma(x_1,x_2)=\lim\limits_{r_1\to 0,r_2\to 0}\frac{E(\vert X\cap U(x_1,r_1) \vert\vert X\cap U(x_2,r_2) \vert)}{\pi^2 r_1^2r_2^2}
\end{equation*}
满足二阶平稳假设时,有$\gamma(x_1,x_2)=\gamma(\vert x_1-x_2 \vert)$。观测到的聚集理论上既可能由于一阶强度非恒定,也可能由于相互作用存在(事件之间不独立);二者很难区分。

\par 完全空间随机模式(Complete Spatial Randomness, CSR)假设:(1)一阶强度恒定,$\lambda(x)\equiv \lambda$;(2)事件之间相互独立,即对任意两个不交的区域$A_1,A_2$,$\operatorname{cov}(\vert X\cap A_1\vert,\vert X\cap A_2\vert)=0$。此时任一子区域$A$的点数量服从均值为$\lambda \vert A \vert$的Poisson分布。

\par \textbf{样方法}:

\par \textbf{核密度估计}:

\section{局部分析}

\par 全局分析是对研究区域的整体概括,可能隐藏空间变化。局部分析强调空间的差异性,其发展动力部分来源于对GIS与空间分析集成的兴趣(局部分析是空间显式的,其结果可用GIS可视化)。

\par \textbf{变量间关系表现出空间非平稳性的原因}:(1)随机采样的变化(\emph{也应包括测量误差等随机波动});并非关注重点,一般关注的是参数估计中的系统变化。(2)关系在空间上的本质差异;自然地理学中少见,但人的态度、偏好和行为是空间变化的,这一观点与后现代主义人类行为研究以位置和局部性为研究框架相符。(3)模型误设,如遗漏变量和函数形式误设;这一观点与实证学派相近,认为行为存在普遍规律,只是我们的模型不够完善。

\par \textbf{局部点模式分析}:点模式的全局统计量给出集聚性、分散性、随机性的总体描述,但无法探测局部异常,而后者在疾病研究中有重要应用。局部点模式分析的进展包括Openshaw提出的地理分析机(Geographical Analysis Machine)及其改进,该方法通过比较实际和期望情形下随机圆形窗口内点的个数识别局部异常。

\par \textbf{局部自相关分析}:可视化方法包括半方差函数云(semivariogram, 变量平方差对距离)和Moran散点图(邻域均值对当前值)。局部自相关指数包括Getis-Ord $G_i$指数和Moran $I_i$指数\footnote{Anselin指出,若要求全局与局部统计量有加和关系,$G_i$不属于局部空间自相关统计量。}。

\par \textbf{局部回归分析}:包括(1)空间自适应滤波(Spatial Adaptive Filtering),参数估计难以进行统计检验。(2)贝叶斯随机系数模型、多层次模型,没有关系空间依赖性的假设。(3)空间扩展法。(4)地理加权回归。

\par 多层次模型同时包含个体因素和地点因素(语境效应),避免仅考虑个体层面的微体谬误(Atomistic Fallacy)和仅考虑总体层面的生态谬误(Ecological Fallacy);需要定义离散的空间单元,且地理过程在边界处发生突变。

\par 空间扩展法的局限性在于需要预先设定方程的形式,且只能建模关系空间变化的整体趋势,可能掩盖局部变化。

\par 地理加权回归原理与局部加权回归(Locally Weighted Regression)、回归参数漂移分析(Drift Analysis of Regression Parameters, DARP)相似;后两种数据权重由属性相似度而非地理距离决定。

\par \textbf{局部空间交互模型}:交互起点为单元$i$的条件下,终点为单元$k$的概率
\begin{equation}
    p_{ik}=\frac{S_k^\alpha d_{ik}^{-\beta}}{\sum_j S_j^\alpha d_{ij}^{-\beta}}
\end{equation}
全局模型对参数$\alpha,\beta$作全局估计,也可对每个起点分别估计模型,得到参数的空间变化。
