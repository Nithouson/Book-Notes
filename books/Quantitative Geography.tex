
\chapter{计量地理学}
\Large\textbf{Quantitative Geography: Perspectives on Spatial Data Analysis}
\par \emph{A.Stewart Fotheringham, Chris Brunsdon, Martin Charlton} \normalsize

\section{概论}
\par 计量地理研究20世纪80年代早期到90年代中期经历低迷,主要在人文地理学领域(自然地理学受影响不大)。其原因包括支撑计量地理学的实证主义哲学的衰落,马克思主义、后现代主义、结构主义、人文主义等新思潮的兴起(地理学具有批判现有范式、追求新范式的特点)。有观点认为计量方法的批评者大多对其了解不够,认为其相对困难。

\par 对计量地理学的批评包括过于注重普适规律,对个体建模不关注认知与行为过程等。作者指出计量地理学的新发展(如局部分析、融入认知和心理过程的空间交互建模)使上述批评不再适用。事实上,自然主义的观点(“地理即物理”,追求普适定律和全局关系),更多被自然地理而非人文地理学者采用。

\par 计量方法提供了认识空间过程的有效、可靠手段。即使对于难以完全量化的人类行为,计量方法也能量化其中可度量的影响因素(如距离阻碍)。

\par 对空间数据特殊性质(\emph{空间效应})的认识是计量地理学发展成熟的标志,计量地理学者由其它学科技术方法的使用者转变为空间数据分析思想的输出者。

\par 写作本书之时计量地理学的进展体现在局部分析方法、探索性数据可视化、地理计算(Geocomputation)等。地理计算强调空间数据定量分析中对计算机计算能力的运用,如Moran's I的显著性检验中,使用t统计量不视为地理计算,Monte Carlo模拟则视为地理计算,后者避免了理论分布可能不满足的问题。
