
\chapter{计量地理学}
\Large\textbf{Quantitative Geography: Perspectives on Spatial Data Analysis}
\par \emph{A.Stewart Fotheringham, Chris Brunsdon, Martin Charlton} \normalsize

\section{概论}
\par 计量地理研究20世纪80年代早期到90年代中期经历低迷,主要在人文地理学领域(自然地理学受影响不大)。其原因包括支撑计量地理学的实证主义哲学的衰落,马克思主义、后现代主义、结构主义、人文主义等新思潮的兴起(地理学具有批判现有范式、追求新范式的特点)。有观点认为计量方法的批评者大多对其了解不够,认为其相对困难。

\par 对计量地理学的批评包括过于注重普适规律,对个体建模不关注认知与行为过程等。作者指出计量地理学的新发展(如局部分析、融入认知和心理过程的空间交互建模)使上述批评不再适用。事实上,自然主义的观点(“地理即物理”,追求普适定律和全局关系),更多被自然地理而非人文地理学者采用。

\par 计量方法提供了认识空间过程的有效、可靠手段。即使对于难以完全量化的人类行为,计量方法也能量化其中可度量的影响因素(如距离阻碍)。

\par 对空间数据特殊性质(\emph{空间效应})的认识是计量地理学发展成熟的标志,计量地理学者由其它学科技术方法的使用者转变为空间数据分析思想的输出者。

\par 写作本书之时计量地理学的进展体现在局部分析方法、探索性数据可视化、地理计算(Geocomputation)等。地理计算强调空间数据定量分析中对计算机计算能力的运用,如Moran's I的显著性检验中,使用t统计量不视为地理计算,Monte Carlo模拟则视为地理计算,后者避免了理论分布可能不满足的问题。

\section{数据探索性可视化}

\subsection{单变量方法}

\par \textbf{茎叶图}.
\par \textbf{箱线图}(boxplot): 五数概括包括最小值$x_{\min}$、1/4分位数$Q_1$、中位数$Q_2$、3/4分位数$Q_3$、最大值$x_{\max}$。箱线图的矩形边界表示上下四分位数,矩形中的横线表示中位数;竖线可延伸到最大、最小值,或延伸到
\begin{equation}
    \min \{x_{\max}, Q_2+1.5(Q_3-Q_1)\}
\end{equation}
及
\begin{equation}
    \max \{x_{\min}, Q_2-1.5(Q_3-Q_1)\}.
\end{equation}
超出此范围的异常值用点标出。

\par\textbf{直方图和频数折线图}:可视为概率密度函数的近似,直方图假设每个分组内概率密度函数为定值,折线图假设为分段线性函数。Terrell和Scott对二者分别给出极大平滑的保守分组个数$(2n)^{1/3}$和$(73.5n)^{1/5}$($n$为观测数),可能存在过度平滑,但更接近真实模式。

\par \textbf{核密度估计}:核$g(\cdot)$是均值为0、方差为1的概率密度函数。对于标量数据,以每个$x_i$为中心生成核,再对所有核求平均得到概率密度函数:
\begin{equation}
    \hat{f}(x)=\frac{1}{n}\sum_{i=1}^n \frac{1}{h}g\left(\frac{x-x_i}{h}\right).
\end{equation}
$h$为带宽,过大则掩盖细节,过小则导致尖峰估计。Terrell给出极大平滑规则:
\begin{equation}
    h \approx s\left(\frac{243\int g^2(x)\text{d} x}{35n}\right)^{1/5}.
\end{equation}
$s$是样本标准差。

\par \textbf{地图}:不同可视化窗口间的关联,特别是数据图与地图的关联很有帮助。

\subsection{多变量方法}

\par 探索的多变量数据特征包括:聚类、异常、趋势。

\par \textbf{散点图矩阵}:$m$个变量的矩阵为$m$行$m$列(对角线为空),同一行具有相同的$y$轴变量,同一列具有相同的$x$轴变量。局限性是只能表示两个变量间的关系。

\par \textbf{平行坐标图}:数据$(x_1,\dots,x_m)$表示为折线$(1,x_1)(2,x_2)\dots(m,x_m)$。实际应用需对各特征标准化处理。可同时表示$m$个变量的信息,但图形模式依赖坐标轴排列顺序(考虑左右对称,$m$个变量有$m!/2$种情形)。

\par \textbf{径向坐标可视化}(Radial Coordinate Visualization, RADVIZ):设$m$个点均匀排布在单位圆周上,位置向量分别为$S_1,\dots,S_m$,数据$(x_1,\dots,x_m)$表示为平面上的点,位置向量为
\begin{equation}
    \mathbf{u}=\frac{\sum_{j=1}^m x_j S_j}{\sum_{j=1}^m x_j}
\end{equation}
其物理解释为:设小球分别通过弹性系数为$x_j$的弹簧与$S_j$相连,则平衡位置为$\mathbf{u}$。

\par 当每个$x_j$非负时,$\mathbf{u}$在$S_1,\dots,S_m$的凸包内。通常对每个变量进行0-1标准化。这一映射当然不是单射,只要每个$x_j$相等,$\mathbf{u}$都取原点。RADVIZ对成分数据(向量各分量之和为1)更适用。考虑旋转和镜像对称,$m$个变量有$(m-1)!/2$种不同的映射;实践中可根据某种目标(如$\mathbf{u}$的方差)做离散搜索优化,也可通过交互控制选取。

\par \textbf{投影寻踪}(Projection Pursuit):指根据要探测的特征选取某个高维特征空间到二(三)维的投影,属于连续优化问题。探测的目标可以是偏离高斯分布的情况(中心孔洞或偏态分布),或最小化平均最邻近距离(Mean Nearest-Neighbor Distance)来探测聚类特征。与之相对的是总体巡查(Grand Tour),投影方向动态变化,以显示所有可能的投影。引导性总体巡查播放投影寻踪中的迭代优化过程。

