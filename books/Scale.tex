
\chapter{规模:复杂世界的简单法则}
\Large\textbf{Scale: The Universal Law of Growth, Innovation, Sustainability, and the Pace of Life in Organisms, Cities, Economies and Companies}
\par \emph{Geoffrey West} \normalsize

\section{规模法则}

\par \textbf{规模缩放法则}:特定量与规模的关系,即系统在规模变化时的响应,也称标度律(scaling law)。常见的关系包括对数线性(幂律)、常值。

\par 设$y=kx^\alpha$,则$\ln y= \alpha \ln x+\ln k$,故幂律指数即取对数回归后的斜率。当$\alpha>1$,称超线性规模缩放,造成规模收益递增,意味着开放式增长;$\alpha<1$,称亚线性规模缩放,造成规模经济,意味着受限生长、有限寿命。

\par 规模法则是支撑不同系统的网络结构优化的结果,源自自然选择、适者生存内在的连续反馈机制。

\par 规模缩放的实例:
\begin{itemize}
    \item Galileo指出物体不能无限按比例扩大,原因是当尺寸增大$k$倍时,横截面的压强也增大$k$倍,无法支撑;除非改变物质组成或结构设计。这意味着Godzilla等虚构形象不会真实存在。实证数据表明,举重冠军成绩与体重呈指数为2/3的幂律,与这一论证一致。
    \item 里氏震级每相差2级,能量相差1000倍。
    \item 随体重线性变化的药物剂量是值得怀疑的;如果药物的吸收是表面积(截面积)起作用,应当按2/3次幂律变化。
    \item 体重与身高平方之比的到的BMI是缺乏概念基础的。这一指数的提出者是Adolphe Quetelet,其提出了“平均人”概念,研究人类的平均生理和社会指标及其分布。
    \item 英国工程师Isambard K. Brunel认为大轮船比小轮船高效,建造“大东方号”效果不佳。Froude发现$\frac{v^2}{lg}$相同时(被称为Froude数,$v$为行驶速度,$l$为船长),船只动力学行为一致。由于大多数规模缩放是非线性的,直接按比例缩放的模型实验常常不具有可靠性。
    
\end{itemize}

\par \textbf{复杂系统的特征}:具备个体组分不具备的集体特性(称为涌现行为),无法简单地从个体行为预测,“整体大于局部之和”;自组织行为;对外部条件的适应性(韧性);混沌和临界现象等。

\section{生命科学}

\par \textbf{生物的标度律}:Kleiber定律指出,动物代谢率与体重呈现3/4次幂规模缩放。动物的寿命随体重1/4次幂增长;但一生的心跳总次数大致为常值(15亿次;人突破到25亿次),与体重无关。主动脉压大致为常值。呼吸频率与心率成正比(约4次心跳对应一次呼吸)。此外,众多标度律的幂次为1/4的倍数(1/4次幂规模法则)。

\par 异速生长(allometric):身形随体型增长发生变化,但变化的比例各不相同,如动物四肢的长度增长比半径慢。

\par \textbf{解释1/4次幂异速生长规模法则}:关注网络特性。假设网络满足空间填充、终端单元恒定性(如哺乳动物有相同直径的毛细血管)、性能优化(耗能最小化)三个特征。循环系统遵循等面积分支,从而脉动区域连续血管半径按$\sqrt{2}$倍率增大。粘滞区域按$\sqrt[3]{2}$倍率增大。连续血管长度按$\sqrt{2}$倍增大。循环网络体积正比于身体体积。综合以上得出1/4次幂规模法则。4对应四个维度,多出的一个源于网络的分形特征。

\par \textbf{哺乳动物的体型极限}:血液从主动脉到毛细血管流动经历脉动、非脉动两个不同模式,而不同体型哺乳动物非脉动血管层级数大致相同。后者更为低效,导致代谢率的线性规模缩放。循环系统至少要包含1级脉动分支,这给出体重下限为数克({\CJKfontspec{simsun.ttc}鼩鼱})。毛细血管间的平均距离遵循1/12次规模缩放,过大使细胞缺氧(氧气扩散的最大距离称为最大Krogh半径),这给出体重上限为100吨(最大的蓝鲸)。

\par \textbf{生物的生长}:代谢率的3/4次规模增长与现有细胞能量需求的线性增长存在1/4次的差值,这导致生长的最终停止。不同动物的无量纲质量与无量纲时间遵循类似的增长曲线,这可以由上述理论预测。

\par \textbf{寿命}:全球平均寿命从1870年的30岁增长到2011年的70岁,主要原因是医疗卫生的发展,以及婴幼儿死亡率的大幅下降。已知的最长寿命为122岁。

\par \textbf{解释寿命随体重1/4次幂增长}:假定死亡的临界条件是受损细胞达到特定比例,而受损细胞数为时间和单位时间受损事件数之积,后者与终端单元数成正比。这推出寿命正比于细胞总数与终端单元数(幂次为3/4)之比,幂次为1/4。

\section{城市科学}

\par 指数增长的特征是增长速度与现有量成正比、倍增时间恒定。当今世界人口倍增时间在缩短,事实上呈现超指数增长。人身体每天消耗的能量相当于100W的功率,而事实上现代生活消耗的能量相当于3000W。

\par 马尔萨斯主义认为,人口指数增长,而资源供应只能线性增长,这导致灾难性的崩溃。创新乐观主义认为,过去200年的高速增长会因人类的天才、创新能力而持续。

\par \textbf{城市}:从物理层面,城市包含道路、电网等网络系统,存在物质、能量和人的流动;从人的层面,城市中的人给城市带来了活力、灵魂和精神,城市被认为是促进社会互动、人类合作和创新的天才机制。

\par 到2006年,世界城市人口已经过半。

\par Jane Jacobs提倡“城市即人”,从公民集体生活的角度看待城市,指出城市规划应以人为主导,而非道路和建筑物。她以保护曼哈顿下城著名,认为城市重建或导致传统社区的摧毁。Ebenezer Howard提出“花园城市”概念,追求城镇和乡村的融合,成为郊区的样板。规划的城市往往充满疏离感、缺少流行和文化活动、缺乏社区精神,但城市具有韧性,“城市会进化,最终发展出灵魂”。

\par 传统的城市理论往往是定性的重点研究,依赖叙事和直觉感知,作者追求定量化、可预测的城市理论,关注城市的共性。

\par \textbf{城市结构}:Walter Christaller提出“中心地理论”,将城市体系描述为六边形晶体结构。事实上城市更像具有分形和褶皱的有机体,其边界线长具有类似于海岸线的分形维数,其增长模式类似于细菌菌落。Michael Batty对城市分形维数进行了广泛研究。其研究侧重社会科学、地理学、城市规划的现象学传统,不同于本书作者的物理学传统。Zipf定律指出城市位序反比于人口规模,即城市人口服从重尾的幂律分布。

\par \textbf{城市标度律}:基础设施指标(道路、管线长度,加油站数量)呈幂律指数0.85的亚线性规模增长,意味着大城市资源节约、污染减少;社会经济指标(总工资、GDP、专利数、餐厅数、犯罪数、病例数)呈幂律指数1.15的超线性规模增长,意味着大城市具有更多的财富、机遇、文娱设施,以及更多的疾病和犯罪。此种关系对同一国家的城市成立,意味着从指标上看,同一国家不同规模的城市是非线性缩放的版本。相对于生物,城市标度律拟合准确程度弱于生物,可能与城市的历史、文化、地理因素有关。

\par 城市排名不应以人均指标为准(隐含线性规模缩放),而应以相比理论预测值的偏移量为准。

\par \textbf{城市标度律的解释}:基础设施网络、社会网络的结构和动力学;资源、能源、信息在其中流动的特性。这两种网络与生物网络有类似的特点:基础设施供应和社会活动定义了城市的边界;房屋和人分别是其不变终端;成本、能源利用的近似最小化(由于历史原因、私人原因可能迟滞)。

\par 两种网络都是自相似分形网络。基础设施网络是分级网络,从一地出发的交通流量体现分形特征。社会网络是“六度分隔”的小世界网络,存在高度互联的模块化小集团;从个体出发,社会网络存在层次结构,相邻层人数具有恒定的比值(Dunbar定律;普通人保持联系的社交网络成员最大数量为150,这称为Dunbar数)。

\par 社会经济指标与人与人之间的连接(互动)成比例。若全连接,幂指数应为2;基础设施、能量利用限制了连接,故实际指数小于2,且前者亚线性缩放的程度与社会经济指标超线性缩放的程度相当。联系生物的1/4规模缩放法则对应四个维度,Luis Bettencourt认为15\%是1/6的近似值。

\par \textbf{生活节奏的加快}:基础设施网络从终端单元起流量逐渐增大,导致亚线性规模缩放和规模经济。与之相反,社会网络从终端起互动、信息交流逐渐减少;社会网络的正反馈机制带来互动、创意、财富的超线性规模增长;与生物生存时间相反,社会经济活动时间根据15\%规模法则压缩。

\par 实证:步行速度随城市人口数增加而加快,幂律指数为1.1. 人均电话联系总次数(反映人与人互动次数)呈现接近1.15的幂律;人均通话总时长也随人口数系统性增加。

\par Marchetti's constant: 源于Y.Zahavi 对每日通勤时间恒定性的观察。每日出行的总时长约为1小时,从而“步行城市”边长约5km,汽车都市圈边长约40km。 这决定了城市规模的上限。

\par 城市人群移动的频率分布:以一定频率访问某一地点的人数与距离成平方反比;距离一定,访问某一地点的人数与访问频率成平方反比。$\rho \sim (rf)^{-2}$

\par \textbf{经济多样性}:实证表明企业数量和雇员总数与人口规模呈线性关系;企业的多样性(类型数)随人口规模以对数缓慢增长。不同类型企业数量在不同城市呈现一致的位序-规模曲线,其理论解释是偏好依附(Yule-Simon过程)的正反馈机制,即企业类型越常见,新增企业属于此类型的概率就越大。随着城市规模增大,超线性增长的企业类型(信息、服务业)占比上升,亚线性增长的企业类型(农业、采矿业)占比下降。

\par \textbf{城市增长}:城市总体代谢的社会经济来源(如财富创造和创新)遵循超线性规模法则,而需求量线性增长,这意味着城市遵循与生物体相反的开放式增长。

\section{公司科学}

\par \textbf{理解公司的传统机制}:交易成本(优化原则,成本最小化、利润最大化);组织架构(公司内的网络结构);市场竞争(进化压力、选择过程)。

\par 基于主体建模(Agent-based modelling):假设主体之间相互作用的简单法则,模拟时序演化。将相互依存的部分解构为半独立子系统会带来误导;主体行为、决策的规则定义具有主观性;主体被视为独特的实体,难以揭示一般性法则。

\par \textbf{公司的标度律}:尽管相较于生物、城市,数据围绕理论曲线的波动更大,其平均水平仍具有幂律模式:总利润、总资产、净收入等随雇员人数亚线性增长。总收入(销售额)呈线性规模增长,这意味着随时间的指数增长(?)

\par \textbf{公司的增长}:公司的代谢率(销售额)线性增长;支出开始时亚线性变化,随规模扩大转为近似线性变化;故增长曲线开始较快,之后增速放缓。

\par \textbf{公司的消亡}:公司的寿命呈现指数分布(恒定死亡率),不论所处的行业或消亡的原因(破产清算、并购)。美国上市公司具有10.5年的半衰期。极端长寿公司(200年以上)的数量超出理论预期,它们大多数规模不大,持续为小众客户群生产高质量产品。

\section{结论}

\par 城市的超线性规模增长意味着随时间的超指数增长,在有限时间内达到无穷大;到达这一奇点时,资源和能量供应的不足将导致停滞和崩溃。对策是在奇点之前重新设定增长参数,这通过创新带来的范式转移实现,但需要范式转移的速度越来越快,即连续创新的间隔不断缩短。呼吁发展“可持续性的大一统理论”。

\par 复杂适应系统的动力学和结构难以用一小部分等式编码。即使个体成分间相互作用的动力学已知,复杂适应系统的细节仍难以预料。本书介绍的标度理论可以理解和预测系统的粗粒度行为。基本粒子及其相互作用的大统一理论不能解释并预测所有事物,因此需要发展“复杂性的大一统理论”。

\par 大数据时代的到来不意味着数据模式挖掘可以节约一切问题,不再需要理解事物的基本原理和因果关系。作者认为,如果没有对事物背后机制的基本思考,很容易误入歧途,浪费大量算力。