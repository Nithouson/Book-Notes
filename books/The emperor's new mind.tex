
\chapter{皇帝新脑}
\Large\textbf{The Emperor's New Mind}
\par \emph{Roger Penrose} \normalsize

\section{机器智能、算法与意识}

\par 什么是思维?什么是精神?它们是否依赖于生物结构?电子设备能否思维?

\par 我们无法逻辑地掌握“精神”的概念,源于对基本物理定律缺乏理解。物理理论不仅在亚原子粒子、大爆炸、相对论与量子理论统一等问题上有空白,与人类思维和意识运行的水平上也存在无知。

\par \textbf{图灵测试}:行为主义观点,即如果电脑的表现和一个人在思维时的表现不能区分,则认为它在思维。具体指以与人类不能区别的方式回答我们关心的问题。对电脑而言,真正困难的问题或许是基于常识、需要理解的问题。

\par 行为主义还宣称AI可以模拟快乐与痛苦。假设机器的行为尽可能增加某一正值,避免其负值,可以定义正值为快乐、负值为痛苦。作者指出缩手等条件反射或许比人对苦乐的反应更接近机器的行为。

\par \textbf{强人工智能}:认为精神活动不过是一种算法,思维、感情、智慧、意识仅仅是头脑所执行算法的特征。反之,如果找到了这样的算法,在机器上运行就可以使机器具有精神。认为只有算法的逻辑结构与其精神状态有关,执行方式与之无关。算法的抽象存在与物理实现完全分离,这导致二元论的立场。

\par 所有现代通用电脑都是相互等价的通用图灵机,这是强人工智能哲学认为硬件相对不重要的依据。但作者认为人的思维可能引起特别的物理现象,且其效应无法用电脑精密地模仿,即“头脑即为电脑”的假设不成立。

\par \textbf{Searle中文屋子}:被恰当编程、根据输入文本正确回答问题的电脑仍不具备“理解”有关的精神属性。假设文本用中文而非英文,一个不懂中文的操作员同样可按电脑程序执行运算,得出正确结果。作者认为论证是有力的,不严格之处在于可能存在与一个人执行算法有关的离体的“理解”,并反射到他的意识。

\par Hofstadrer假想一本包含Einstein头脑全部描述的书,可以像Einstein本人一样回答问题(根据图灵测试,它就是Einstein)。强人工智能将认为其具有思维、理解和知觉。作者质疑,书何以知觉自己被提问?是否当书中算法的激活或改变(包括其存储空间的改变)时其知觉才被唤起?

\par 一个人的个性和自我认知与其组成成分无关(同种基本粒子是全同的),而是体现在这些成分的排列模式。强人工智能进一步认为,这一排列模式的信息被无损地翻译为另一种形式时,人的个性保持不变,其意识和知觉继续存留。这引出科幻中“超距传送”的设想,即以电磁信号束发送人的完整信息,在目的地重新装配。这是旅行还是复制?如果出发地的人不销毁,知觉会同时存在两个地方吗?