
\chapter{皇帝新脑}
\Large\textbf{The Emperor's New Mind}
\par \emph{Roger Penrose} \normalsize

\section{机器智能、算法与意识}

\par 什么是思维?什么是精神?它们是否依赖于生物结构?电子设备能否思维?

\par 我们无法逻辑地掌握“精神”的概念,源于对基本物理定律缺乏理解。物理理论不仅在亚原子粒子、大爆炸、相对论与量子理论统一等问题上有空白,与人类思维和意识运行的水平上也存在无知。

\par \textbf{图灵测试}:行为主义观点,即如果电脑的表现和一个人在思维时的表现不能区分,则认为它在思维。具体指以与人类不能区别的方式回答我们关心的问题。对电脑而言,真正困难的问题或许是基于常识、需要理解的问题。

\par 行为主义还宣称AI可以模拟快乐与痛苦。假设机器的行为尽可能增加某一正值,避免其负值,可以定义正值为快乐、负值为痛苦。作者指出缩手等条件反射或许比人对苦乐的反应更接近机器的行为。

\par \textbf{强人工智能}:认为精神活动不过是一种算法,思维、感情、智慧、意识仅仅是头脑所执行算法的特征。反之,如果找到了这样的算法,在机器上运行就可以使机器具有精神。认为只有算法的逻辑结构与其精神状态有关,执行方式与之无关。算法的抽象存在与物理实现完全分离,这导致二元论的立场。

\par 所有现代通用电脑都是相互等价的通用图灵机,这是强人工智能哲学认为硬件相对不重要的依据。但作者认为人的思维可能引起特别的物理现象,且其效应无法用电脑精密地模仿,即“头脑即为电脑”的假设不成立。

\par \textbf{Searle中文屋子}:被恰当编程、根据输入文本正确回答问题的电脑仍不具备“理解”有关的精神属性。假设文本用中文而非英文,一个不懂中文的操作员同样可按电脑程序执行运算,得出正确结果。作者认为论证是有力的,不严格之处在于可能存在与一个人执行算法有关的离体的“理解”,并反射到他的意识。

\par Hofstadrer假想一本包含Einstein头脑全部描述的书,可以像Einstein本人一样回答问题(根据图灵测试,它就是Einstein)。强人工智能将认为其具有思维、理解和知觉。作者质疑,书何以知觉自己被提问?是否当书中算法的激活或改变(包括其存储空间的改变)时其知觉才被唤起?

\par 一个人的个性和自我认知与其组成成分无关(同种基本粒子是全同的),而是体现在这些成分的排列模式。强人工智能进一步认为,这一排列模式的信息被无损地翻译为另一种形式时,人的个性保持不变,其意识和知觉继续存留。这引出科幻中“超距传送”的设想,即以电磁信号束发送人的完整信息,在目的地重新装配。这是旅行还是复制?如果出发地的人不销毁,知觉会同时存在两个地方吗?

\par Church-Turing论题认为,图灵机(或$\lambda$-演算等等价概念)定义了算法,即我们认为的机械的逻辑或数学运算。利用Cantor对角线法可以证明停机问题($HALT_{TM}$)不是可判定的,且可以找到一个算法,对任何可识别$HALT_{TM}$的算法(接受停机实例,对不停机实例可能拒绝或不停机),给出一个不停机实例使之不停机。可计算性决定了图灵机计算能力的理论边界。那么物理系统(包括人脑)的计算能力与图灵机相比如何?可计算数/问题是否足以描述实在的物理对象(可计算数只有可数多)?

\section{数学、实在与证明}
\par 实数系统与物理概念并非在所有尺度下都一致。如实数的值可以无限小,但距离和时间间隔不一定(量子引力尺度下通常的距离概念或不再有意义)。

\par 数学上的发明更适合称为``发现'',它们不仅是头脑的创造物,而是可以具有某种抽象的绝对性和实在性,这一观点被称为``\textbf{数学柏拉图主义}''。例如Mandelbrot集的发现,以及复数的引入(尽管引进负数的平方根似乎是一种工具式的发明,但从中获取的远比最初设计的多得多)。进一步地,作者认为数学真理的绝对性和数学概念的柏拉图存在性在本质上等同。

\par 数学的\textbf{形式主义}主张抽去数学陈述中的意义,只将其视为形式化的符号串。哥德尔不完全性定理给了形式主义毁灭性的打击,即一个相容的形式系统必存在不能证明的真命题。连续统假设在ZF公理体系下不可判定,形式主义者无法讨论其真伪。

\par \textbf{哥德尔定理的证明思路}:形式数学系统定义了有限的符号表(包括数字、变量、运算符、逻辑符号、量词等),故其符号串可数。特别地,考虑含单个自然数变量的命题函数的序列$P_n(w)$和证明的序列$\Pi_n$。命题$\lnot \exists x [\Pi_x \text{证明} P_w(w)]$同样可用系统编码,设其为$P_k(w)$。故我们有
\begin{displaymath}
\lnot \exists x [\Pi_x \text{证明} P_k(k)]=P_k(k)
\end{displaymath}
若$P_k(k)$为假,则其存在证明,矛盾。故$P_k(k)$为真,且不存在证明。

\par 可以将哥德尔命题$P_k(k)$作为公理加入系统,系统又会出现新的哥德尔命题。得出哥德尔命题为真需要形式系统之外的直觉。事实上数学家对定理的证明也需要超越形式系统步骤的洞察。哥德尔定理表明仅在建立形式系统时考虑符号的实际意义是不够的;数学真理的概念不能包容在形式主义框架之中。

\par 此外,停机问题不可解性与哥德尔不完全性有对应:形式系统中验证一个符号串是否证明给定命题是可计算的。对于完备的系统,由于所有证明可列,可依序检查所有证明的结果,从而存在可以判定任何命题真伪的图灵机(命题本身及其否定,恰有一个是某个证明的结论)。可识别$HALT_{TM}$的算法无法识别的实例与哥德尔命题有对应,体现了洞察对形式运算的超越。

\par 形式系统中可证明的命题是递归可枚举集;但其补集不是(考虑命题$S(n)$:编号为$n$的图灵机对输入$n$停机,若不然,所有假的$S(n)$是递归可枚举的,这与停机问题不可解矛盾)。由于可证伪的命题也是递归可枚举的,表明有命题既不可证明有不可证伪。真命题和假命题的集合都不是递归可枚举的(同样由于假的$S(n)$不是递归可枚举的)。

\par \textbf{直觉主义}:以拓扑学家Luitzen E.J. Brouwer为代表,只接受具有确定构造的数学对象的存在性,认为没有实在构造的存在没有意义。不接受排中律($\lnot (\lnot P) \iff P$) 和反证法(如假设某对象不存在,推出谬误,以此证明存在)。这一观点导致数学论证中有力工具的丧失。事实上基于图灵可计算性研究可构造性,可将其与存在性分开讨论。

\par 对于非递归集而言,不存在判定元素是否属于此集合的算法。Mandelbrot集的补集是递归可枚举的,作者猜想其本身不是\footnote{原则上递归集、递归可枚举集只能用于可数集,这里的讨论只考虑单个点的判定,而且可能需要排除边界(考虑单位圆盘,判定点在圆盘内/外都是可计算的,但由于判定两个可计算数相等是不可计算的,判定点是否在单位圆盘的边界上也是不可计算的)。}。此外,丢番图方程组解的存在性是不可判定的;半群的字问题不仅对于一般等价表不可判定,对特定等价表也可能是不可判定的;一组多边形是否能镶嵌平面也是不可判定的\footnote{作者给出了两种多边形只能非周期镶嵌平面的例子,即Penrose铺陈。}。

\section{经典和量子物理}
\par 作者认为现有对物理的理解不足以描述大脑的运行。牛顿力学和相对论构成的经典物理是决定性的,而量子力学是非决定性的理论。作者认为量子现象对大脑运行是重要的。