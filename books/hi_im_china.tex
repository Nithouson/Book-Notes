
\chapter{这里是中国}

\subsection*{中国从哪里来?}

约6500万年前印度、亚欧板块碰撞产生了青藏高原,深刻影响了中国地貌、气候、水文格局。
\begin{itemize}
\item 地貌:云贵高原、黄土高原、内蒙古高原受挤压影响抬升,形成第二级阶梯(三级阶梯:从荒原到都市;人地关系)
\item 气候:北纬30度副热带高压带本应干旱,青藏高原夏季受热更多产生上升气流(?),抽吸季风,加剧东南季风深入大陆,导致江南的湿润。南亚季风被阻挡,形成塔克拉玛干沙漠;盛行西风被阻挡,携带西北沙尘形成黄土高原。
\item 水系:“高原水塔”,大量冰川,江河发源地
\item 生物:许多北极动物发源于青藏高原(因适应寒冷,能在260万年前大冰期生存)
\end{itemize}

\subsection*{可可西里:中国最伟大的荒野}

北侧昆仑山,南侧唐古拉山,中央可可西里山;地势相对平缓的沉积盆地;火山遗迹、温泉群\\
90\%永久冻土,80-120米厚;大量湖泊、沼泽\\
植物:簇生柔子草(垫状植物,先锋)\\
动物:鼠兔,旱獭,黑颈鹤,盘羊,藏原羚,藏野驴,野牦牛,藏羚羊(雄性打斗求偶,主要繁殖地:卓乃湖;往返迁徙),藏狐,狼\\
1994 杰桑•索南达杰 反藏羚羊盗猎牺牲\\
2017 入选世界遗产,核心区、缓冲区6万平方千米

\subsection*{阿里:荒野文明}

西藏海拔最高的区域\\
山峰:纳木那尼7694(喜马拉雅)、冈仁波齐6656(冈底斯山脉主峰)\\
湖泊:扎日南木错(西藏第三大湖)、玛旁雍错\\
札达土林:2400平方千米,湖盆沉积物被流水切割\\
文明:象雄王国(前4世纪-7世纪):土林上的都城:大鹏银城;雍仲本教发源:不分方向的万字符(雍仲符号)作为神圣符号\\
古格王国(9世纪-17世纪):吐蕃瓦解,后裔建政权,有“阿里”之称(意为:领地);引入印度佛教;被拉达克入侵者所灭

\subsection*{横断山:中国极致风光最密集的山脉}

南北走向七山脉(横断七脉)平均间距100km\\
山峰:梅里雪山6740(他念他翁山-怒山);玉龙雪山5596(芒康山-云岭);雀儿山6168(沙鲁里山脉);格聂山6204(沙鲁里山脉);贡嘎山7556(大雪山);四姑娘山(幺妹峰)6250(邛崃山脉);雪宝顶5588(岷山)\\
河流:怒江、澜沧江、金沙江、雅砻江、大渡河、岷江(横断六江)。怒江、澜沧江峡谷:高差达3000-4000米,向南流出国境;金沙江:丽江石鼓镇长江第一湾 向东接纳后三江
若尔盖湿地\\
黄龙、九寨沟:钙华景观(流水溶解碳酸盐,沿途沉积)\\
丰富的垂直自然带

\subsection*{九寨沟:毁灭与创造}

地质基础:浅海古生物骨骼形成的碳酸盐岩沉积\\
冰川侵蚀产生陡峭山峰、U型谷\\
崩塌、滑坡物形成瀑布、堰塞湖;钙华沉积也会形成瀑布台基、分隔湖泊\\
钙华的不同颜色与其表面微生物种类有关

\subsection*{四姑娘山:冰与岩之歌}

冰川侵蚀、流水改造:U形谷-V形谷;冰斗湖、冰瀑\\
丰富的花岗岩:富含Na、K,利于植物生长。长成乔木的沙棘;红石(藻类附着花岗岩)\\
花岗岩峰群:多为尖削的角峰;五色山:向斜构造;大峰、二峰、三峰:入门级山峰;幺妹峰:攀登困难,1981年首登

\subsection*{伊犁:遥远西部的一个角落}

天山主脉、支脉形成向西的喇叭口,携带大西洋水汽的西风形成大量降水(迎风坡600-800mm)\\
伊犁河:新疆水量最大的河流,向西注入巴尔喀什湖\\
植被:雪岭云杉林、野生果树;草原\\
龚自珍:“北可以制南,南不可制北”(伊犁可以支撑草原政权,南疆绿洲城邦势力弱小)\\
乾隆帝进军经营伊犁:造城、屯田、移民

\subsection*{罗布泊:楼兰生死五千年}

位于塔里木盆地最低洼处,孔雀河、车尔臣河、塔里木河曾经汇集成湖;后下游断流,湖泊干涸,成为遍布荒漠的死亡地带\\
雅丹地貌分布区\\
前3000年吐火罗人进入、定居\\
小河墓地:船形棺木(胡杨木);“小河公主”等干尸\\
楼兰:前2世纪-6世纪,汉时设西域长史府,前77年改名鄯善,东汉中期统一塔里木盆地东南部,成为东西方文化交流通道;因自然环境恶化消失\\
尼雅遗址:“五星出东方利中国”织锦;{\CJKfontspec{simsun.ttc}佉}卢文书:“我渴望追求文学、音乐以及天地间一切知识”\\
20世纪后半叶对塔里木盆地的开发使罗布泊完全干涸\\
1901 斯文•赫定发现楼兰古城\\
1972 Landsat-1拍摄“地球之耳”\\
1980 彭加木失踪\\
1996 探险家余纯顺脱水而亡

\subsection*{甘肃:愈多元愈美丽}

省名取自甘州(张掖)、肃州(酒泉)\\
三大自然区交会处;多元宗教:崆峒山,拉卜楞寺,莫高窟,麦积山石窟\\
陇东、陇中黄土高原:沟壑顶部平地(黄土塬)易耕种\\
陇南山地:秦岭、岷山交会\\
河西走廊:干旱;祁连山东段降雨丰富,有草原、冰川,中国第二大内陆河黑河发源;“河西四郡”位于绿洲之上\\
甘南高原:青藏高原边缘

\subsection*{西安:鲜衣怒马一千年}

关中平原,北依黄土高原,南临秦岭;秦岭水系带来大量泥沙,厚沉积物

\subsection*{成都:烟火人间三千年}

成都平原:龙门山、龙泉山之间,河流泥沙沉积\\
前4世纪建城,城名未改,城址未迁移;约3000年前 古蜀国(金沙遗址);秦灭古蜀 李冰修都江堰;蜀设锦官城(蜀锦的官府作坊);张献忠屠城;元末以后大量移民涌入;杨森修森威路(今春熙路),成中央商务区

\subsection*{梵净山:红尘孤岛}

主体为变质岩,不同于周围碳酸盐岩,不易被侵蚀;降水充沛\\
最高峰凤凰山2570\\
新金顶上部有裂隙,一分为二\\
孑遗植物:鹅掌楸、珙桐\\
黔金丝猴的最后栖息地

\subsection*{河南:造山、造水、造中华}

西北太行山:断层形成,河南一侧有陡峭悬崖(林州大峡谷)-安阳\\
西部秦岭余脉:熊耳山、伏牛山-洛阳\\
南侧:桐柏山、大别山-南阳\\
中部、东部:平原-开封

\subsection*{浙江:无敌生产力}

以丘陵山地为主,“七山一水二分田”;岛屿最多的省份\\
得名于钱塘江(又称浙江、之江,自上游依次称新安江、富春江、钱塘江)\\
黄茅尖1929(洞宫山脉);神仙居(括苍山脉);雁荡山:流纹火山岩\\
历史:河姆渡(约7000年前):水稻;良渚(约5300年前):玉饰、丝绸;越人(春秋):越剑、鸟虫书;明清工商业市镇猛增(西塘、南浔、乌镇)

\subsection*{福建:开拓者传奇}

80\%以上山地、丘陵,沿海平原;年均降水量超2000mm,河流众多\\
武夷山:赣闽边界;丹霞地貌。太姥山:花岗岩峰林\\
闽江:省内最长河流\\
四大平原:福州(闽江)、莆田(木兰溪)、泉州(晋江)、漳州(九龙江)\\
支离破碎地形导致文化孤岛(不同方言、戏剧、神灵、民俗)\\
元代泉州成为重要港口,出口武夷岩茶等物产

\subsection*{青岛:城市美学史}

1897-1914 德国占领青岛,规划建造城市。以总督府为中心;规定建筑外形不能重复;城市绿化的样板(红瓦绿树)\\
1931-1935 八大关别墅群;圣弥爱尔大教堂;里院的大规模发展(四周合围,中心成院)

\subsection*{江南:江河湖海的盛宴}
文化地理概念:长江、钱塘江下游沿岸以及太湖流域\\
南京:东晋、南北朝,国都\\
扬州:隋唐,运河枢纽\\
杭州:南宋,国都\\
苏州:明清,纺织业,工商业\\
上海:清末以来,港口,工商业

\subsection*{什么是中国?}
农业模式:南稻北稷(包括粟和黍),后北方被小麦玉米等取代