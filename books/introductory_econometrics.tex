\chapter{计量经济学导论:现代观点}
\Large\textbf{Introductory Econometrics: A Modern Approach}
\par \emph{Jeffrey M. Wooldridge} \normalsize

\section{导论}

\par 计量经济学是数理统计的一个分支,主要研究\textbf{非实验经济数据}收集与分析的固有问题。非实验数据(non-experimental data)也称观测数据(observational data),区别于(自然科学中)控制实验得到的数据,强调研究者只是被动的数据收集者。

\par 计量模型中自变量的识别依据经济理论或直觉。\textbf{误差项}(error term)或干扰项(disturbance term)包含了不可观测及未识别的变量,以及度量自变量时的误差。

\par 数据集分类:(1)\textbf{横截面数据}(cross-sectional data), 同一时间段(或不考虑时间差异)多个样本,通常假定是随机抽样得到(即样本独立同分布)。(2)\textbf{时间序列数据}(time series data),一个或多个变量不同时间的观测值,一般不能假定观测独立于时间,需考虑自相关、季节效应等。(3)\textbf{混合横截面数据}(pooled cross-section data),多个时间段横截面数据的组合(每个时间段对应不同随机样本)。(4)\textbf{面板数据}(panel data),多个样本的时间序列,区别于混合横截面数据,每个单元都被重复观测;可以控制观测单元无法观测的特征,有助于因果推断和时间滞后的研究。

\par 研究两个变量的关系时往往需要其它(相关)因素不变(ceteris paribus);退一步的要求是所关注自变量的选择独立于其它相关因素,从而可以视为实验数据。非实验数据的特征导致难以达到识别因果效应(casual effect)的目标。

\section{横截面数据的线性回归模型}

\subsection{模型假定}
\par \textbf{(1)线性于参数}:总体模型为
\begin{equation}
    y=\beta_0+\beta_1 x_1+\dots+\beta_k x_k +u
\end{equation}
其中$\beta_0$称截距(intercept),$\beta_j(j=1,\dots,k)$称斜率参数(slope parameter),$u$为误差项。
\par 这一模型称为多元线性回归(multiple linear regression)模型;当$k=1$时,称为简单线性回归(simple linear regression)模型。

\par \textbf{(2)随机抽样}:设有来自上述总体模型的含$n$个观测的随机样本$\{(x_{i1}, \dots, x_{ik}, y_i)\}_{i=1}^N$,也即每个样本满足
\begin{equation}
    y_i=\beta_0+\beta_1 x_{i1}+\dots+\beta_k x_{ik} +u_i.
\end{equation}

\par \textbf{(3)不存在完全共线性}:样本中没有一个自变量是常数,自变量之间也没有严格线性关系。
\par 这一假定保证了普通最小二乘法(ordinary least squares, OLS)估计的存在唯一性。假定的成立必要条件是样本容量$n\ge k+1$。自变量之间的非线性依赖关系是被允许的。

\par \textbf{(4)零条件均值}:给定自变量任何值,误差期望为0.
\begin{equation}
    E(u\vert x_1, \dots, x_k)=0.
\end{equation}
\par 这一假定保证了OLS参数估计的无偏性。自变量与误差独立是一个更强的条件。违背此假定的情形包括因变量与自变量关系的误设(如应为二次函数但模型未包括二次项)、遗漏变量(且该变量与任一自变量相关)等。
\par 假定(1)(4)可写成
\begin{equation}
    E(y\vert x_1,\dots,x_k)=\beta_0+\beta_1 x_1+\dots+\beta_k x_k \label{eq:prf}
\end{equation}
这称为总体回归函数(population regression function)。

\par \textbf{(4')零均值和零相关}:对$j=1,2,\dots,k$,
\begin{equation}
    Eu=0, \text{Cov}(x_j,u)=0
\end{equation}
(4)蕴含(4'). 这一假设与OLS一阶条件对应。(1)-(3),(4')即可得到OLS估计量的一致性,但不能保证无偏性;且得不到总体回归函数(\ref{eq:prf})。若某个自变量$x_j$与$u$相关,称为\textbf{内生解释变量}(endogenous explanatory variable)。

\par \textbf{(5)同方差性(homoskedasticity)}:给定自变量任何值,误差方差相同,
\begin{equation}
    \text{Var}(u\vert x_1, \dots, x_k)=\sigma^2
\end{equation}
也即$\text{Var}(y\vert x_1, \dots, x_k)=\sigma^2$。误差方差随自变量变化,称为异方差性(heteroskedasticity)。
\par 假设(1)-(5)称为横截面数据回归的Gauss-Markov假定。

\par \textbf{(6)正态性}:总体误差$u$独立于解释变量$x_1,\dots,x_k$,且服从均值为0、方差为$\sigma^2$的正态分布。
\par 这一假设蕴含(4)(5),用于得到OLS斜率估计值的抽样分布,从而进行假设检验和置信区间估计。该假定的依据可以来自中心极限定理,但要求误差是众多独立同分布随机因素的和(事实上各因素总体分布也可以不一致,但会影响正态近似的程度)。

\par (1)-(6)合称为经典线性模型(classical linear model, CLM)假设。CLM总体假定可表述为
\begin{equation}
    y\vert x_1,\dots x_k \sim N(\beta_0+\beta_1x_1+\dots+\beta_kx_k,\sigma^2)
\end{equation}

\subsection{最小二乘法}

\par 给定自变量、因变量的一个样本,被估计的方程,即样本回归函数(sample regression function)形如
\begin{equation}
    \hat{y}=\hat{\beta}_0+\hat{\beta}_1 x_1+\dots+\hat{\beta}_k x_k
\end{equation}
其中$\hat{\beta}_0$称截距估计值(intercept estimate);$\hat{\beta}_j(j=1,\dots,k)$称斜率估计值(slope estimate),代表了其它变量不变时$x_j$对$y$的偏效应(partial effect)。观测$i$的拟合值为
\begin{equation}
    \hat{y}_i=\hat{\beta}_0+\hat{\beta}_1 x_{i1}+\dots+\hat{\beta}_k x_{ik}
\end{equation}
残差(residual)定义为
\begin{equation}
    \hat{u}_i = y_i-\hat{y}_i.
\end{equation}
区别于不可观测的误差,残差是可以计算得到的。普通最小二乘法最小化残差平方和,得到如下一阶条件(first order conditions):
\begin{align}
    &\sum_{i=1}^N \hat{u}_i=0\\
    &\sum_{i=1}^N \hat{u}_ix_{ij}=0, j=1,\dots,k
\end{align}
这也可以由零条件均值假设推出:上式为$Eu=0, E(x_ju)=0$在样本中的对应。

\par 斜率估计值的一个公式为
\begin{equation}
    \hat{\beta}_j=\frac{\sum_{i=1}^n \hat{r}_{ij}y_i}{\sum_{i=1}^n \hat{r}_{ij}^2}
\end{equation}
其中$\hat{r}_{ij}$为$x_j$对$x_1,\dots,x_{j-1},x_{j+1},\dots,x_k$的回归残差。右端即$y$对该残差进行简单线性回归的斜率估计,表明多元回归度量了“排除其他变量影响”时的关系。

\par 定义总平方和(total sum of squares, SST), 解释平方和(explained sum of squares, SSE), 残差平方和(sum of squared residuals, SSR)如下:
\begin{align}
    \text{SST}&=\sum_{i=1}^n (y_i-\bar{y}^2)\\
    \text{SSE}&=\sum_{i=1}^n (\hat{y}_i-\bar{y}^2)\\
    \text{SSR}&=\sum_{i=1}^n u_i^2
\end{align}
可以证明SST=SSR+SSE。定义决定系数(coefficient of determination,R-squared,$R^2$):
\begin{equation}
    R^2=1-\text{SSR}/\text{SST}\label{eq:Rsquared}
\end{equation}
$R^2$同时是实际值$y_i$和拟合值$\hat{y}_i$相关系数的平方。注意这只对于线性回归成立,在其它模型下,二者可能不同;按式(\ref{eq:Rsquared})计算,$R^2$可能为负;但按相关系数平方计算,$R^2$总在[0,1]中。

\par 在方程中增加一个自变量,通常总会使$R^2$增大;调整$R^2$(adjusted R-squared)对自变量个数进行了惩罚:
\begin{equation}
    \bar{R}^2=1-\frac{\text{SSR}/(n-k-1)}{\text{SST}/(n-1)}\label{eq:adjustedR2}
\end{equation}
事实上,增加一个(一组)自变量时,调整$R^2$增加当且仅当其$t$统计量($F$统计量)大于1。此定义的依据是,设总体$R^2$为$1-\sigma_u^2/\sigma_y^2$,即总体中$y$的变化可被自变量解释的部分;(\ref{eq:adjustedR2})第二项的分子、分母分别是$\sigma_u^2,\sigma_y^2$的无偏估计。但事实上调整$R^2$并不是一个无偏估计量。调整$R^2$也可能为负。对于两个非嵌套模型(non-nested model,即没有一个是另一个的特殊情况),可用调整$R^2$进行模型选择。

\subsection{OLS估计量的期望、方差}

\par 在假定(1)-(4)下,OLS估计量是总体参数的无偏估计量,即
\begin{equation}
    E(\hat{\beta}_j)=\beta_j, j=0,1,\dots,k
\end{equation}
\par 在假定(1)-(5)下,OLS斜率估计量以自变量为条件的方差为
\begin{equation}
    \text{Var}(\hat{\beta}_j)=\frac{\sigma^2}{\text{SST}_j (1-R_j^2)}, j=0,1,\dots,k \label{eq:Variance}
\end{equation}
其中$SST_j=\sum_{i=1}^n (x_{ij}-\bar{x}_j)^2$,$R_j^2$是$x_j$对所有其它自变量回归得到的$R^2$。
\par 在假定(1)-(5)下,
\begin{equation}
    \hat{\sigma}^2=\text{SSR}/(n-k-1)
\end{equation}
是误差方差的无偏估计。其平方根$\hat{\sigma}$称回归标准误(standard error of regression, SER),\textbf{不是}误差标准差的无偏估计。OLS斜率估计量的标准差(standard deviation)为
\begin{equation}
    \text{sd}(\hat{\beta}_j)=\sigma/\sqrt{\text{SST}_j(1-R_j^2)}
\end{equation}
其估计值(称$\hat{\beta}_j$的标准误)为:
\begin{equation}
    \text{se}(\hat{\beta}_j)=\hat{\sigma}/\sqrt{\text{SST}_j(1-R_j^2)}
\end{equation}
随着样本量的增大,其以$1/\sqrt{n}$速率收敛到0. 

\subsection{Gauss-Markov定理}

\par 在假定(1)-(5)下,对任何$j=0,1,\dots,k$,设$\Tilde{\beta}_j$是$\beta_j$的任一无偏估计量,且具有形式$\Tilde{\beta}_j=\sum_{i=1}^n w_{ij}y_i$,其中$w_{ij}$是自变量值的函数,则有
\begin{equation}
    \text{Var}(\Tilde{\beta}_j)\ge \text{Var}(\hat{\beta}_j)
\end{equation}
也即OLS估计量是最优(方差最小)的线性无偏估计量(best linear unbiased estimate)。
\par 进一步,在假设(1)-(6)下,OLS估计量是最小方差无偏估计量(无需限定线性)。


\subsection{总体参数的区间估计和假设检验}

\subsubsection{t检验}
\par 在假定(1)-(6)下,OLS斜率估计量抽样分布为正态分布:
\begin{equation}
    (\hat{\beta}_j-\beta_j)/\text{sd}(\hat{\beta}_j) \sim N(0, 1)
\end{equation}
\par 用标准误估计标准差,得到$t$分布:
\begin{equation}
    (\hat{\beta}_j-\beta_j)/\text{se}(\hat{\beta}_j) \sim t_{n-k-1}
\end{equation}
随着自由度增加,$t$分布会接近标准正态分布。当自由度大于120,可以使用正态分布的临界值。

\par 总体参数$\beta_j$的\textbf{置信区间}为
\begin{equation}
    [\hat{\beta}_j-c\times \text{se}(\hat{\beta}_j),\hat{\beta}_j+c\times\text{se}(\hat{\beta}_j)]
\end{equation}
当置信水平为$\alpha$时,$c$为对应$t$分布的$(1+\alpha)/2$分位数。

\par 设原假设$H_0: \beta_j=a_j$(最常见的情形是$a_j=0$),$t$统计量定义为$(\hat{\beta}_j-a_j)/\text{se}(\hat{\beta}_j)$。双侧备择假设(two-sided alternative)$H_1: \beta_j\neq a_j$对应双边检验(two-tailed test),单侧备择假设(one-sided alternative)$H_1: \beta_j > a_j$(或$H_1: \beta_j < a_j$)对应单边检验(one-tailed test)。若拒绝原假设$H_0: \beta_j=0$,称$x_j$在给定显著性水平下是统计显著的。检验的$p$值定义为能拒绝原假设的最小显著性水平,即原假设成立时,统计量(双边检验时为统计量的绝对值)不小于观测值的概率。

\par 除统计显著性外,还要关注变量的经济(实际)显著性(economic/practical significance),即OLS斜率估计值的大小。大样本下容易得到统计显著性,即使斜率估计值代表的实际含义影响很小。
\par 检验多个总体参数的线性关系,如$\beta_1=\beta_2$,可通过变量代换$\beta_1=\beta_2+\theta_1$,转化为检验$\theta_1$的统计显著性。

\subsubsection{F检验}
\par 检验总体参数的多个线性关系,称原模型为不受约束模型(unrestricted model),代入多重约束后的模型为受约束模型(restricted model),定义$F$统计量为
\begin{equation}
    F=\frac{(\text{SSR}_r-\text{SSR}_{ur})/q}{\text{SSR}_{ur}/(n-k-1)}\sim F_{q,n-k-1}
\end{equation}
其中$\text{SSR}_{ur}$,$\text{SSR}_{r}$分别为原模型和受约束模型的残差平方和,$q$为线性约束的个数,也即两模型的自由度之差。
\par 一种常见情形是$q$个排除性约束:$H_0:\beta_{k-q+1}=0,\dots,\beta_{k}=0$,此时若拒绝原假设,称$x_{k-q+1},\dots,x_{k}$在给定显著性水平下联合显著。此时$F$统计量改写为
\begin{equation}
    F=\frac{(R_{ur}^2-R_{r}^2)/q}{(1-R_{ur}^2)/(n-k-1)}
\end{equation}
当$q=k$时,即$H_0:\beta_1=\dots=\beta_k=0$,有
\begin{equation}
    F=\frac{R^2/k}{(1-R^2)/(n-k-1)}
\end{equation}
这称为回归的整体显著性(overall significance of the regression)检验。

\par 检验单个变量排除性假设的$F$统计量等于相应$t$统计量的平方,而$t_{n-k-1}^2$与$F_{1,n-k-1}$同分布,故二者等价。但对一组变量而言,可能出现每个变量均不显著但联合显著(如由于多重共线性),或单个显著变量与若干不显著变量的整体联合不显著。

\subsubsection{Lagrange乘数检验}

\par Lagrange乘数统计量(Lagrange multiplier statistic)也用于检验多个排除约束。设$H_0:\beta_{k-q+1}=0,\dots,\beta_{k}=0$。先估计受约束模型,得到残差$\Tilde{u}$,再将$\Tilde{u}$对所有自变量做回归,记其$R^2$为$R_u^2$。Lagrange乘数统计量定义为:
\begin{equation}
    LM=nR_u^2
\end{equation}
$n\to \infty$时依分布收敛于自由度为$q$的$\chi^2$分布。若LM大于卡方分布临界值,则拒绝原假设。

\subsection{OLS估计量的渐近性质}

\par OLS估计量的无偏性、有效性(Gauss-Markov定理)等属于有限样本性质(或精确性质);渐近性质(asymptotic properties, 或大样本性质)关注样本容量趋于无穷大的情形。经济学界认为一致性($n\to \infty$时依概率收敛到总体参数)是对估计量的基本要求。

\par 在假定(1)-(3),(4')下,OLS估计量$\hat{\beta}_j$是${\beta}_j$的\textbf{一致估计量}, $j=0,1,\dots,k$. 事实上对于简单线性回归,斜率估计量的
概率极限\begin{equation}
    \text{plim}\hat{\beta}_1=\beta_1+\frac{\text{Cov}(x_1,u)}{\text{Var}(x_1)}
\end{equation}
右端第二项是误差与自变量相关导致的渐近偏误(asymptotic bias)。对于一般情形,与OLS斜率估计量期望的偏差类似,若$x_1$与$u$相关,其余变量与$x_1$和$u$都不相关,则只有$\hat{\beta}_1$不一致。

\par 在假定(1)-(5)下,OLS估计量满足\textbf{渐近正态性}(asymptotic normality)。当$n \to \infty$,
\par (a) $\sqrt{n}(\hat{\beta}_j-\beta_j)$依分布收敛到$N(0,\sigma^2/a_j^2)$, 其中$a_j^2=\text{plim}(\sum_{i=1}^n \hat{r}_{ij}^2/n)$,$\sigma^2/a_j^2$称为$\sqrt{n}(\hat{\beta}_j-\beta_j)$的渐近方差;
\par (b) $\hat{\sigma}^2$是$\sigma^2=\text{Var}(u)$的一致估计量,从而$\hat{\sigma}$是$\sigma$的一致估计量;
\par (c) $(\hat{\beta}_j-\beta_j)/\text{sd}(\hat{\beta}_j)$依分布收敛到$N(0,1)$;$(\hat{\beta}_j-\beta_j)/\text{se}(\hat{\beta}_j)$依分布收敛到$N(0,1)$.
\par 这意味着大样本条件下无需假设(6),OLS的置信区间估计、t检验、F检验可如常进行。

\par 在假定(1)-(5)下,OLS估计量满足\textbf{渐近有效性}。考虑求解如下方程组得到的估计量:
\begin{equation}
    \sum_{i=1}^n g_j(x_{i1},\dots,x_{ik})(y_i-\Tilde{\beta}_0-\Tilde{\beta}_1x_{i1}-\dots-\Tilde{\beta}_kx_{ik})=0, j=0,1,\dots,k
\end{equation}
其中$g_j$是对自变量的任意函数,可以证明由此得到的估计量是一致的。在所有上述形式的估计量中,OLS估计量的$\sqrt{n}(\hat{\beta}_j-\beta_j)$具有最小的渐近方差。

\subsection{预测区间估计}

\par 设自变量取值$x_1=c_1,\dots,x_k=c_k$,令$\theta_0=\beta_0+\beta_1 c_1+\dots+\beta_k c_k$,则有$\theta_0=E(y\vert x_1=c_1,\dots,x_k=c_k)$。估计量$\hat{\theta}_0=\hat{\beta}_0+\hat{\beta}_1 c_1+\dots+\hat{\beta}_k c_k$给出$\theta_0$的无偏估计,其标准误由$y$对$x_1-c_1,\dots,x_k-c_k$回归截距项的标准误得到,由此可计算条件均值$\theta_0$的置信区间。

\par 实际用于预测时,还需纳入误差的影响。设$y_0=\beta_0+\beta_1 c_1+\dots+\beta_k c_k+u_0$,估计量$\hat{y}_0=\hat{\theta}_0$,预测误差
\begin{equation}
    \hat{e}_0=\beta_0+\beta_1 c_1+\dots+\beta_k c_k+u_0-\hat{y}_0
\end{equation}
以样本内自变量取值为条件,其期望为0,方差为$\text{Var}(\hat{y}_0)+\sigma^2$。定义预测误差的标准误
\begin{equation}
    \text{se}(\hat{e}_0)=\sqrt{(\text{se}(\hat{y}_0))^2+\hat{\sigma}^2}
\end{equation}
在假定(1)-(6)下,$\hat{e}_0/\text{se}(\hat{e}_0)$服从$n-k-1$个自由度的$t$分布,由此可计算预测区间(prediction interval)。

\subsection{过原点回归}
\par 若利用最小二乘法估计不含截距项的模型,则残差的样本均值不一定为0;若$\beta_0\neq 0$,OLS的斜率估计量将有偏。若$\beta_0=0$,带截距回归的OLS斜率估计量方差比过原点回归大。

\subsection{非线性关系建模}
\par \textbf{常弹性模型}:$\ln y=\beta_0+\beta_1 \ln x$,解读为$x$每增加1\%,$y$增加$\beta_1$\%,$\beta_1$称为$y$对$x$的弹性。

\par \textbf{常半弹性模型}:$\ln y=\beta_0+\beta_1 x$,解读为$x$每增加1,$y$增加100$\beta_1$\%,100$\beta_1$称为$y$对$x$的半弹性。

\par 当模型为$\ln y=\beta_0+\beta_1 x_1+\dots+\beta_k x_k+u$时,由$\ln y_0$的预测区间上下界$[c_l,c_m]$可得$y_0$的预测区间$[e^{c_l},e^{c_m}]$;但点估计不能直接通过指数运算得到,在假定(1)-(6)下,
\begin{equation}
    E(y\vert \mathbf{x})=\exp(\sigma^2/2)\exp(\beta_0+\beta_1 x_1+\dots+\beta_k x_k)
\end{equation}
故预测值应调整为
\begin{equation}
    \hat{y}=\exp(\hat{\sigma}^2/2)\exp( \hat{\ln y})
\end{equation}
这一估计量一致但非无偏。若不满足正态性假设,预测值为
\begin{align}
&E(y\vert \mathbf{x})=\alpha_0\exp(\beta_0+\beta_1 x_1+\dots+\beta_k x_k)\\
&\hat{y}=\hat{\alpha}_0\exp( \hat{\ln y})
\end{align}
$\alpha_0$的矩估计量是
\begin{equation}
    \hat{\alpha}_0=\frac{1}{n}\sum_{i=1}^n \exp(\hat{u}_i)
\end{equation}
这一估计量一致但非无偏,且总大于1(除非所有残差为0)。
\par 令$m=\exp(\beta_0+\beta_1 x_1+\dots+\beta_k x_k)$,则$E(y\vert m)=\alpha_0 m$。记$\hat{m}_i=\exp (\hat{\ln y_i})$,利用过原点回归得到$\alpha_0$的估计量
\begin{equation}
    \check{\alpha}_0=\frac{\sum_{i=1}^n \hat{m}_i y_i}{\sum_{i=1}^n \hat{m}_i^2}
\end{equation}
这一估计量一致但非无偏,且可能小于1(此时可能意味着假定不满足)。
\par 因变量为对数时的$R^2$不能和因变量为水平值的情形直接比较,解决方法有二:一是计算上述$\hat{m}_i$与$y_i$相关系数的平方(不依赖$\alpha_0$估计量的选取);二是计算
\begin{equation}
    1-\frac{\sum_{i=1}^n(y_i-\hat{\alpha}_0\exp(\hat{\ln y_i}))^2}{\sum_{i=1}^n(y_i-\bar{y})^2}
\end{equation}

\par 当方程包含二次项或两个变量的乘积项时,变量的偏效应取决于自变量取值。平均偏效应(average partial effect)即自变量取样本均值时的偏效应。包含乘积项时称这两个变量具有\textbf{交互效应}(interaction effect);通过将二次项改写为$x_1(x_2-c_2)$,则$x_1$的一次项系数可反映$x_2=c_2$时$x_1$的偏效应,同时可得到其标准误。

\subsection{遗漏变量偏误}

\par 若包含了无关变量,即总体模型中事实上某个$\beta_j=0$,其它自变量斜率估计值的无偏性不受影响。

\par 若遗漏了有关变量,则所有OLS估计量都可能存在偏误,称遗漏变量偏误(omitted variable bias)。不失一般性,设遗漏变量$x_k$后的OLS估计量为$\Tilde{\beta}_j,j=0,1,\dots,k-1$,$x_k$对$x_1,\dots,x_{k-1}$回归的斜率估计值为$\Tilde{\delta}_j,j=1,\dots,k-1$,则有
\begin{equation}
    \Tilde{\beta}_j=\hat{\beta}_j+\hat{\beta}_k\Tilde{\delta}_j
\end{equation}
其关于自变量的条件期望为
\begin{equation}
    E(\Tilde{\beta}_j)=\beta_j+\beta_k\Tilde{\delta}_j
\end{equation}
故除非$x_k$是无关变量,或回归系数$\Tilde{\delta}_j=0$。有时可以利用$\beta_k$和$\Tilde{\delta}_j$的符号推测遗漏变量偏误的方向。

\par 考虑遗漏一个变量后的简单线性回归,真实模型为$y=\beta_0+\beta_1x_1+\beta_2x_2+v$,满足假定(1)-(4),若遗漏变量$x_2$,则有
\begin{align}
    &E(\Tilde{\beta}_1)=\beta_1+\beta_2\Tilde{\delta}_1\\ &\Tilde{\delta}_1=\frac{\sum_{i=1}^n (x_{i1}-\bar{x}_1)(x_{i2}-\bar{x}_2)}{\sum_{i=1}^n (x_{i1}-\bar{x}_1)^2}
\end{align}
而渐近极限
\begin{align}
    &\text{plim}\Tilde{\beta}_1=\beta_1+\beta_2\delta_1\\ &\delta_1=\frac{\text{Cov}(x_1,x_2)}{\text{Var}(x_1)}
\end{align}
右式第二项为遗漏变量带来的不一致性。$\Tilde{\delta}_1$代表$x_1,x_2$的样本相关性,而$\delta_1$代表它们在总体中的相关性。

\par 对于OLS估计量的方差而言,遗漏变量会使其减小(\ref{eq:Variance})。但方差会随样本量的增大而减小,而期望的偏差不会减小(期望会越来越接近渐近极限),故大样本情况下更倾向于使用完整模型,而不是为避免多重共线性加剧遗漏变量。

\subsection{多重共线性}

\par 多重共线性(multicollinearity)指自变量之间高度相关(但不完全相关)。多重共线性不影响OLS估计量的无偏性,但会使其方差增大。
\par 方差膨胀因子(variance inflation factor, VIF)定义为:
\begin{equation}
    \text{VIF}=1/(1-R_j^2)
\end{equation}
不应依据$R^2_j$或VIF值简单推断多重共线性的影响,因为OLS估计量的标准差还取决于误差方差和自变量的样本波动。此外若关注的变量与出现多重共线性的变量不相关,多重共线性不影响关注变量的方差。

\subsection{模型误设}

\par \textbf{邹至庄(Chow)检验}:用于检验组间回归方程的差异,是F检验的一种特殊形式。设分组模型为:
\begin{equation}
    y=\beta_{g,0}+\beta_{g,1}x_1+\dots+\beta_{g,k}x_k+u
\end{equation}
其中$g=1,2$。两方程可合写为包含1个虚拟变量、$k$个虚拟变量交互项的一个方程,自由度为$2k+2$。考虑对假设$H_0: \beta_{1,j}=\beta_{2,j},j=0,\dots,k$的F检验:
\begin{equation}
    F=\frac{(\text{SSR}_P-\text{SSR}_1-\text{SSR}_2)/(k+1)}{(\text{SSR}_1+\text{SSR}_2)/(n-2k-2)}
\end{equation}
其中$\text{SSR}_P$是两组样本合并估计方程的残差平方和。

\subsection{关于实践的评注}
\par 为得到以标准差为单位度量的偏效应,可先将所有自变量和因变量进行Z-score标准化(减去均值,除以标准差),此时得到的系数$\hat{b}_j$称为标准化系数,其与未标准化系数的关系是
\begin{equation}
    \hat{b}_j=\frac{\hat{\sigma}_j}{\hat{\sigma}_y}\hat{\beta}_j
\end{equation}

\par 对取值为正的变量取对数,往往可以缓和偏态性和异方差性,且变量的取值范围变小,估计值受异常值影响更小。

\par 较低的$R^2$并不意味着无法得到准确的偏效应估计。只要未加入模型的解释因素与自变量无关(假定(4)成立),就不影响OLS斜率估计量的无偏性;样本量的增大也可抵消误差方差对估计量方差的影响。但较低的$R^2$会给预测带来困难。

\par 若关注$x_1$对$y$的影响,而这种影响通过中间变量$x_2$实现,就不应将$x_2$加入模型。如吸烟对家庭医疗支出的影响,不应将去医院次数作为解释变量。判定依据是“其它条件不变”是否有意义。但对于影响$y$又与$x_1$无关的自变量,总可以将其加入模型以减少估计量的方差。

\par 定性信息可作为二值变量(也称虚拟变量,binary/dummy variable)加入模型;对于$m$类分类变量(分级等序数量类似处理),一般选其中一个为基组(base group),加入$m-1$个虚拟变量。自变量中的虚拟变量度量不同分组的截距差异;虚拟变量与非虚拟变量的交互项度量不同分组的斜率差异。若因变量为二值变量,称为线性概率模型(linear probability model),斜率系数度量了单位自变量增加导致的响应概率$P(y=1)$的变化,其缺陷是可能得到小于0或大于1的概率拟合值,且异方差性往往存在:
\begin{equation}
    \text{Var}(y\vert \textbf{x})=p(\textbf{x})(1-p(\textbf{x}))
\end{equation}
其中$p(\textbf{x})=\beta_0+\beta_1 x_1+\dots+\beta_kx_k$为成功概率。