\chapter{千年难题}
\Large\textbf{The Millennium Problems: The Seven Greatest Unsolved Mathematical Puzzles of Our Time}
\par \emph{Keith Devlin} \normalsize

\par 千年难题奖金由企业家Landon Clay捐赠,由其设立的非营利性组织克雷数学促进会(Clay Mathematics Institute)掌管,七个问题各提供100万美元奖金。

\par \textbf{黎曼猜想}:1859年黎曼在一篇论文中提出。除本身不够准确而没有确定答案的问题外,黎曼猜想是1900年希尔伯特问题中唯一到2000年仍未被解决的。欧拉的$\zeta$函数定义为
\begin{equation}
    \zeta(s)=\sum_{n=1}^\infty \frac{1}{n^s}=\prod_{p\, \text{prime}}\frac{1}{1-(\frac{1}{p})^s}, s>1
\end{equation}
其包含素数分布的信息。黎曼将其解析延拓到除$z=1$外的复平面,得到的复值函数称为黎曼$\zeta$函数;并发现该函数的零点与素数的密度函数$\#\{p<N\vert p\, \text{prime}\}/N$相关。该函数在所有负偶数取值为0;黎曼猜想断言其余零点的实部均为1/2. 目前已用计算机验证了猜想对(按模长升序排列的)前15亿个零点成立。