\chapter{千年难题}
\Large\textbf{The Millennium Problems: The Seven Greatest Unsolved Mathematical Puzzles of Our Time}
\par \emph{Keith Devlin} \normalsize

\par 千年难题奖金由企业家Landon Clay捐赠,由其设立的非营利性组织克雷数学促进会(Clay Mathematics Institute)掌管,七个问题各提供100万美元奖金。

\par \textbf{Riemann猜想}:1859年黎曼在一篇论文中提出。除本身不够准确而没有确定答案的问题外,黎曼猜想是1900年希尔伯特问题中唯一到2000年仍未被解决的。欧拉的$\zeta$函数定义为
\begin{equation}
    \zeta(s)=\sum_{n=1}^\infty \frac{1}{n^s}=\prod_{p\, \text{prime}}\frac{1}{1-(\frac{1}{p})^s}, s>1
\end{equation}
其包含素数分布的信息。黎曼考虑其复值函数形式,并将其解析延拓到除$z=1$外的复平面,得到的函数称为黎曼$\zeta$函数;并发现该函数的零点与素数的密度函数$\#\{p<N\vert p\, \text{prime}\}/N$相关。该函数在所有负偶数取值为0;黎曼猜想断言其余零点的实部均为1/2. 目前已用计算机验证了猜想对(按模长升序排列的)前15亿个零点成立。

\par \textbf{Yang-Mills方程存在性与质量缺口}:相对论和量子力学的矛盾呼唤物质的大统一理论,其可以解释引力、电磁力、强核力、弱核力。大多数寻找大统一理论的努力在于发展量子场论。将麦克斯韦电磁理论扩展为量子场论的努力产生了规范理论,这一理论给时空中的每一点指派一个对称群。这一方向的理论突破包括量子电动力学(QED)、电弱理论、量子色动力学(QCD)等。Yang-Mills方程组是麦克斯韦方程组的量子理论版本。QED是电磁理论的量子描述,而Yang-Mills方程组同时具有经典性和量子性,故是前者的扩展。该方程组尚无通解公式,题目要求证明对任何紧、单的规范群,四维欧氏空间中的量子Yang-Mills方程组有解,且该解预言质量缺口(即不存在无质量的粒子波)。

\par \textbf{P对NP问题}:大整数质因数分解是NP的,故P=NP将威胁到RSA密码体系的安全。作者认为可能的解决路径是构造一个自然不存在多项式时间解法的NP问题。

\par \textbf{Navier-Stokes方程存在性与光滑性}:Navier-Stokes方程描述了黏性流体在外力作用下的运动:
\begin{align}
    &\operatorname{div} \mathbf{u} =0\\
    &\frac{\partial \mathbf{u}}{\partial t}+(\mathbf{u}\cdot  \nabla)\mathbf{u}=\mathbf{f}-\operatorname{grad} p + v \nabla^2 \mathbf{u}
\end{align}
其中$\mathbf{u}(x,y,z,t)$为流体速度,$\mathbf{f}(x,y,z,t)$为外力,$p(x,y,z,t)$为流体压强,$v$为黏度(流体内部摩擦力的度量)。欧拉给出了无摩擦流体的方程,Navier-Stokes方程在其基础上增加了黏度项。二维情形下方程可解,但三维情形下解的存在性未知。

\par \textbf{Poincaré猜想}:闭曲面的拓扑分类早已解决,利用Euler示性数和可定向性也可以区分任何两个闭曲面。庞加莱猜想指出,若一个三维闭流形是单连通的(基本群是平凡的,任一闭曲线可连续收缩为一点),则它同胚于三维超球面。20世纪数学家证明了庞加莱猜想对四维或更高维流形成立。

\par \textbf{Birch和Swinnerton-Dyer猜想}:设$E$是一条椭圆曲线,$E(\mathbb{Q})$是其上有理点的集合(含无穷远点),其可定义Abel群结构(两点连线与$E$交点关于$x$轴对称点),Louis Mordell证明了$E(\mathbb{Q})$是有限生成的,从而可表为有限阶元素构成的子群与$\mathbb{Z}^r$的直积,$r$被称为$E$的Mordell-Weil秩。Birch和Swinnerton-Dyer猜想指出椭圆曲线的Mordell-Weil秩等于其L函数在1处的零点阶数。

\par \textbf{Hodge猜想}:一个非奇异射影代数簇上的每个(一定类型的)调和微分形式都是代数闭链上同调类的有理组合。调和微分形式是Laplace方程的解;上同调类是闭微分形式依据差的恰当性进行的等价分类。

