
\chapter{Nature Methods方法简介}

\section{Association, correlation and causation}
\par ``Correlation implies association, but not causation. Causation implies association, but not
correlation.''
\par 关联(Association)与依赖(Dependence)同义,指变量之间不独立。有因果关系一定不独立,但不一定表现出相关性(如$y=x^2$)。相关性是关联的一种,不能推出因果是由于可能存在混淆变量(与自变量、因变量都相关)。
\par Pearson相关系数度量线性相关关系,Spearman相关系数度量增减趋势。\emph{后者也可以用在间隔量、比率量,即用排名做Pearson相关,这比原始数据的Pearson相关更宽松。} Pearson相关系数的统计显著性用t分布做检验;也可以利用方差分析得出自变量解释了因变量多大部分的变异。



