\chapter{素数之恋} \label{chap:rh}
\Large\textbf{Prime Obsession: Bernhard Riemann and the Greatest Unsolved Problem in Mathematics}
\par \emph{John Derbyshire} \normalsize

\par 黎曼假设来源于Bernhard Riemann在1859年因当选柏林科学院通讯院士提交的论文《论小于一个给定值的素数个数》。欧拉定义$\zeta$函数为
\begin{equation}
    \zeta(s)=\sum_{n=1}^\infty \frac{1}{n^s}=\prod_{p\, \text{prime}}\frac{1}{1-(\frac{1}{p})^s}, s>1
\end{equation}
第二个等号称为欧拉积公式。黎曼考虑其复值函数形式$\zeta(z)$,其在$\text{Re}z>1$收敛;并将其解析延拓到$\mathbb{C}/\{z=1\}$,得到的函数称为黎曼$\zeta$函数;该函数在所有负偶数取值为0,称平凡零点;另有无穷多个非平凡零点位于$0<\text{Re}z<1$的临界带内,且关于实轴、关于临界线$\text{Re}z=1/2$对称出现。Hardy于1914年证明了位于临界线上的非平凡零点有无穷多个。黎曼假设断言非平凡零点均在临界线上。目前已用计算机验证了猜想对(按模长升序排列的)前15亿个零点成立;另一方面,E. Artin, A. Weil, P. Deligne在有限域上发展了类似于黎曼假设的结果。
\par 定义素数的计数函数$\pi(x)=\#\{p\le x\vert p\, \text{prime}\}$,再令
\begin{equation}
    J(x)=\sum_{n=1}^\infty \frac{1}{n}\pi(x^{\frac{1}{n}}), x>0
\end{equation}
注意对给定的$x$,$J(x)$为有限项的和。由Möbius反演得
\begin{equation}
    \pi(x)=\sum_{n=1}^\infty \frac{\mu(n)}{n}J(x^{\frac{1}{n}})
\end{equation}
其中$\mu(n)$是Möbius函数。黎曼将欧拉积公式表达为积分形式:
\begin{equation}
    \frac{1}{s}\ln\zeta(s)=\int_0^\infty J(x)x^{-s-1}\mathrm{d}x
\end{equation}
这导向黎曼1859年论文的主要结果(由von Mangoldt于1895年证明)
\begin{equation}
    J(x)=Li(x)-\sum_\rho Li(x^\rho)-\ln 2 +\int_x^\infty \frac{1}{t(t^2-1)\ln t}\mathrm{d}t
\end{equation}
其中$Li(x)=\int_0^x \frac{1}{\ln t}\mathrm{d}t$, $\rho$取遍$\zeta(z)$的所有非平凡零点,即$\pi(x)$与$\zeta(z)$的非平凡零点密切相关。事实上,1896年Hadamard和de la Vallée Poussin独立证明素数定理$\pi(x)\sim Li(x)$,即依赖黎曼假设的弱化结果:非平凡零点满足$\text{Re} z<1$。1901年von Koch证明,如果黎曼假设成立,那么有
\begin{equation}
    \pi(x)=Li(x)+O(\sqrt{x}\ln x)
\end{equation}
事实上该命题与黎曼假设等价。基于
\begin{equation}
    \frac{1}{\zeta(s)}=\sum_{n=1}^\infty \frac{\mu(n)}{n^s}
\end{equation}
黎曼假设的另一等价命题是
\begin{equation}
    \sum_{k=1}^n\mu(k)=O(n^{1/2+\epsilon}), \forall \epsilon>0
\end{equation}
\par Littlewood于1914年证明随x增大,$Li(x)-\pi(x)$由正变负,再由负变正,由此反复无穷多次(尽管第一次由正变负的位置非常大,目前计算达不到)。
