\chapter{科学之路:人、机器与未来}
\Large\textbf{Quand la machine apprend}
\par \emph{杨立昆}(\emph{Yann LeCun}) \normalsize

\par 信息的自由流动就是(科学技术)进步的动力。——杨立昆

\par \textbf{学术生涯}:作者在学生时代开始研究多层神经网络的训练,提出了分层学习机 (Hierarchical Learning Machine,HLM)。HLM仍采用输出为±1的二值神经元,但其可反向传播训练参数的思想与Rumelhart, Hinton的工作类似,属于梯度反向传播的“某种奇怪版本”。
\par 作者在Hinton实验室、贝尔实验室工作期间发明了卷积神经网络,用于邮政编码以及手写支票的识别。后者受到福岛邦彦研究的启发,他提出的“认知机”(Cognitron)已具备接近CNN的结构,但只能训练最后一层参数。神经生物学家Hubel和Wiesel对视觉系统的研究启发了福岛邦彦和作者的工作。
\par 语言学界对“人类的语言机制是否是先天的”有争论。Noam Chomsky认为大脑生来具有让人们学习说话的结构;而Jean Piaget认为这一结构也是后天习得的,智力是人与外界交流学习的结果,这启发了作者对机器学习的思考。
\par 作者从未怀疑过神经网络的有效性:人类智能如此复杂,必须建立一个具有自我学习能力和经验学习能力的自组织才能复制它。作者坚信深度学习就是人工智能的未来。

\par \textbf{关于学术研究}:“我贪婪地阅读,我熟知前人的所有工作。在探索之路上我们并非孤身一人,时机到来之时,那些已经存在但尚未提出的理念会一个接一个地涌入许多人的头脑中。”

\par \textbf{深度学习的应用}:社交媒体网站上不良内容的识别(尚不完善,如机器无法理解语言中的隐含意思或反讽、难以区分色情内容和严肃艺术)、用户是否点击广告的预测;人脸识别、图片搜索用到基于嵌入向量的相似度度量;根据说话者的嵌入向量,语音克隆技术将单词(音素)序列转换为说话者的语音。

\par \textbf{人工智能的现状}:深度神经网络运作复杂、难以进行数学分析,且无法进行逻辑推理。机器智能高度专业化、没有常识和意识。人脑比机器更全面、更具可塑性。相同运算能力的GPU与人脑相比,功率大致高6个数量级。
\par “人工智能的世界日新月异,不断地挑战新的极限。当一个关键问题被攻破后,便会进军新的领域,旧的领域便不再属于人工智能的范畴,而是会作为惯用工具存在。”

\par \textbf{前景与挑战}:监督学习依赖大量带标签数据,且会受对抗攻击影响;强化学习不需要给系统提供正确答案,只需要评估系统给出答案的质量,但即使对于简单任务也需要大量交互,用于现实任务(如自动驾驶)需要精确的模拟器。作者认为动物和人类学习的主要方式——自监督学习才是智能的本质\footnote{当“被遮盖的部分”无法确定地预测时,单词等简单输出的不确定性可用概率分布向量表示,而图像等高维输出可用带隐变量的模型(如Conditional GAN)建模。}。人类主要通过观察学习世界运行的规律,并建立预测模型(知晓行为的后果);不具备此种预测能力的强化学习系统只能进行大量试错。

\par 机器应当能像人类和动物一样有效地学习,通过自监督学习获得世界模型,并积累起足够多的世界知识以产生某些常识。作者设想的“带预测模型的自主智能体”包括感知模块、世界模型(根据状态、动作和可能的隐变量预测下一状态)、评价函数\footnote{这里的评价函数是机器自己计算的,不同于强化学习中外界给出的奖惩。};且可以衡量预测的不确定性,确保模型在可靠区域使用。

\par 作者认为不必苛求人工智能系统的可解释性,人类往往将在测试中表现出足够可靠性的技术投入使用,即使未深入了解其运行机制;但应用于司法、医疗等对个人有重大影响的决策时需要给出合理的解释。

\par 物理学理论常常归结为公式和少量自由参数,而其它一些自然科学和社会科学中的复杂现象无法如此简化。当元素相同且细节无关紧要时,可采用热力学和统计物理的方法;而当元素不同、相互作用复杂、细节重要时,只能建立唯象模型。目前的机器学习仅限于此;基于人工智能的因果推理是一个前沿方向。

\par 作者认为人类的大脑是可以计算的生化机器,原则上可以通过计算机再现,尽管有反对者认为人类还不够了解生物、物理、量子等系统如何在人体内结合以使大脑发挥作用。

\par \textbf{《让历史告诉未来》(黄铁军序)}:智能是系统通过获取和加工信息获得的能力,环境是智能的真正来源。大数据是当前智能系统成功的根本因素(小样本方法也是以大数据预训练为前提),计算资源和学习算法主要影响效率。更全面的环境模型(数字孪生)可能哺育更强的人工智能。
\par 人工智能早期由符号主义主导,但符号描述和逻辑推理不是智能的基础,只是一种表现;人类存在不必或不能符号化的“潜智能”。\emph{我更倾向于认为符号与亚符号并列,也存在不借助符号难以精确表达的形式化知识。}机器学习使人工智能的重心从人工赋予智能转移到机器自行习得智能。连接主义和行为主义发展出的深度学习和强化学习是当今人工智能的两大主要方法。
\par 不同于杨立昆的观点,作者认为强化学习是比自监督学习更基本的学习方法;人类自监督学习的能力依赖大脑,机器也需要相应的可塑载体(如深度神经网络);而强化学习对载体的要求更低,对温度敏感的有机大分子就能进行强化学习。“结构决定功能”,相比于学习方法,从事学习的机器载体同样重要。自然智能是大自然训练优化的结果,可探索结构逼近生物神经网络的类脑智能(神经形态计算)。