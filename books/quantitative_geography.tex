
\chapter{计量地理学}
\Large\textbf{Quantitative Geography: Perspectives on Spatial Data Analysis}
\par \emph{A. Stewart Fotheringham, Chris Brunsdon, Martin Charlton} \normalsize

\section{概论}
\par 计量地理研究20世纪80年代早期到90年代中期经历低迷,主要在人文地理学领域(自然地理学受影响不大)。其原因包括支撑计量地理学的实证主义哲学\footnote{实证主义(positivism)认为所有的真知都可通过观察、实验、逻辑推理等科学方法证实。}的衰落,马克思主义、后现代主义、结构主义、人文主义等新思潮的兴起(地理学具有批判现有范式、追求新范式的特点)。有观点认为计量方法的批评者大多对其了解不够,认为其相对困难。

\par 对计量地理学的批评包括过于注重普适规律,对个体建模不关注认知与行为过程等。作者指出计量地理学的新发展(如局部分析、融入认知和心理过程的空间交互建模)使上述批评不再适用。事实上,自然主义的观点(“地理即物理”,追求普适定律和全局关系),更多被自然地理而非人文地理学者采用。

\par 计量方法提供了认识空间过程的有效、可靠手段。即使对于难以完全量化的人类行为,计量方法也能量化其中可度量的影响因素(如距离阻碍)。

\par 对空间数据特殊性质(\emph{空间效应})的认识是计量地理学发展成熟的标志,计量地理学者由其它学科技术方法的使用者转变为空间数据分析思想的输出者。

\par 写作本书之时计量地理学的进展体现在局部分析方法、探索性数据可视化、地学计算(Geocomputation)等。地理计算强调空间数据定量分析中对计算机计算能力的运用,如Moran's I的显著性检验中,使用t统计量不视为地理计算,Monte Carlo模拟则视为地理计算,后者规避了理论分布可能不满足的问题。

\section{数据探索性可视化}

\subsection{单变量方法}

\par \textbf{茎叶图}.
\par \textbf{箱线图}(boxplot): 五数概括包括最小值$x_{\min}$、1/4分位数$Q_1$、中位数$Q_2$、3/4分位数$Q_3$、最大值$x_{\max}$。箱线图的矩形边界表示上下四分位数,矩形中的横线表示中位数;竖线可延伸到最大、最小值,或延伸到\footnote{也有时延伸到10\%分位数和90\%分位数}
\begin{equation}
    \min \{x_{\max}, Q_2+1.5(Q_3-Q_1)\}
\end{equation}
及
\begin{equation}
    \max \{x_{\min}, Q_2-1.5(Q_3-Q_1)\}.
\end{equation}
超出此范围的异常值用点标出。

\par\textbf{直方图和频数折线图}:可视为概率密度函数的近似,直方图假设每个分组内概率密度函数为定值,折线图假设为分段线性函数。Terrell和Scott对二者分别给出极大平滑的保守分组个数$(2n)^{1/3}$和$(73.5n)^{1/5}$($n$为观测数),可能存在过度平滑,但更接近真实模式。

\par \textbf{核密度估计}:核$g(\cdot)$是均值为0、方差为1的概率密度函数。对于标量数据,以每个$x_i$为中心生成核,再对所有核求平均得到概率密度函数:
\begin{equation}
    \hat{f}(x)=\frac{1}{n}\sum_{i=1}^n \frac{1}{h}g\left(\frac{x-x_i}{h}\right).
\end{equation}
$h$为带宽,过大则掩盖细节,过小则导致尖峰估计。Terrell给出极大平滑规则:
\begin{equation}
    h \approx s\left(\frac{243\int g^2(x)\text{d} x}{35n}\right)^{1/5}.
\end{equation}
$s$是样本标准差。

\par \textbf{地图}:不同可视化窗口间的关联,特别是数据图与地图的关联很有帮助。

\subsection{多变量方法}

\par 探索的多变量数据特征包括:聚类、异常、趋势。

\par \textbf{散点图矩阵}:$m$个变量的矩阵为$m$行$m$列(对角线为空),同一行具有相同的$y$轴变量,同一列具有相同的$x$轴变量。局限性是只能表示两个变量间的关系。

\par \textbf{平行坐标图}:数据$(x_1,\dots,x_m)$表示为折线$(1,x_1)(2,x_2)\dots(m,x_m)$。实际应用需对各特征标准化处理。可同时表示$m$个变量的信息,但图形模式依赖坐标轴排列顺序(考虑左右对称,$m$个变量有$m!/2$种情形)。

\par \textbf{径向坐标可视化}(Radial Coordinate Visualization, RADVIZ):设$m$个点均匀排布在单位圆周上,位置向量分别为$S_1,\dots,S_m$,数据$(x_1,\dots,x_m)$表示为平面上的点,位置向量为
\begin{equation}
    \mathbf{u}=\frac{\sum_{j=1}^m x_j S_j}{\sum_{j=1}^m x_j}
\end{equation}
其物理解释为:设小球分别通过弹性系数为$x_j$的弹簧与$S_j$相连,则平衡位置为$\mathbf{u}$。

\par 当每个$x_j$非负时,$\mathbf{u}$在$S_1,\dots,S_m$的凸包内。通常对每个变量进行0-1标准化。这一映射当然不是单射,只要每个$x_j$相等,$\mathbf{u}$都取原点。RADVIZ对成分数据(向量各分量之和为1)更适用。考虑旋转和镜像对称,$m$个变量有$(m-1)!/2$种不同的映射;实践中可根据某种目标(如$\mathbf{u}$的方差)做离散搜索优化,也可通过交互控制选取。

\par \textbf{投影寻踪}(Projection Pursuit):指根据要探测的特征选取某个高维特征空间到二(三)维的投影,属于连续优化问题。探测的目标可以是偏离高斯分布的情况(中心孔洞或偏态分布),或最小化平均最邻近距离(Mean Nearest-Neighbor Distance)来探测聚类特征。与之相对的是总体巡查(Grand Tour),投影方向动态变化,以显示所有可能的投影。引导性总体巡查播放投影寻踪中的迭代优化过程。

\section{点模式分析}
\par 点模式分析主要检测点集空间分布的两种现象:聚集和分散,即聚集程度大于或小于随机分布;比较同一区域多组点集分布间的关系,或点集与参考场间的关系(如病例相对于人口密度的集中程度)。

\par \textbf{探索性分析}:绘制空间散点图(注意检查是否有重合点,如有可改用气泡图);计算点集重心,标准距离(到重心的均方根距离)\footnote{当点集不止一个聚簇时,重心和标准距离可能意义不大。};
\begin{equation}
    d_s = \sqrt{\frac{1}{n}\sum_{i=1}^n (x_i-\bar{x})^2+(y_i-\bar{y})^2}
\end{equation}
计算最邻近距离(Nearest-Neighbor Distance)的分布。

\par \textbf{点过程强度}:设$X$是研究区$R$内的点集,$\mathbf{x}$点处的强度定义为
\begin{equation*}
\lambda(\mathbf{x})=\lim\limits_{r\to 0}\frac{E(\vert X\cap U(\mathbf{x},r) \vert)}{\pi r^2}
\end{equation*}
其与概率密度差一个常数:
\begin{equation}
f(\mathbf{x})=\frac{\lambda(\mathbf{x})}{E(\vert X\cap R \vert)}
\end{equation}
考虑不同位置的相互作用,定义二阶强度为
\begin{equation*}
\gamma(\mathbf{x}_1,\mathbf{x}_2)=\lim\limits_{r_1\to 0,r_2\to 0}\frac{E(\vert X\cap U(\mathbf{x}_1,r_1) \vert\vert X\cap U(\mathbf{x}_2,r_2) \vert)}{\pi^2 r_1^2r_2^2}
\end{equation*}
满足二阶平稳假设时,有$\gamma(\mathbf{x}_1,\mathbf{x}_2)=\gamma(\mathbf{x}_1-\mathbf{x}_2)$。观测到的聚集理论上既可能由于一阶强度非恒定,也可能由于相互作用存在(事件之间不独立);二者很难区分。因此一阶模型假定事件之间相互独立,强度非均质;二阶模型假定强度均质,但事件间非独立。两类方法对于单一点集的集聚性分析结果往往一致,但对于两个点集的比较,可能一阶属性(核密度曲面)差异显著而二阶属性(K函数)差异不显著。

\par 完全空间随机模式(Complete Spatial Randomness, CSR)假设:(1)一阶强度恒定,$\lambda(\mathbf{x})\equiv \lambda$;(2)事件之间相互独立,即对任意两个不交的区域$A_1,A_2$,$\operatorname{cov}(\vert X\cap A_1\vert,\vert X\cap A_2\vert)=0$。此时任一子区域$A$的点数量服从均值为$\lambda \vert A \vert$的Poisson分布。

\par 一阶强度分析方法包括样方法、核密度估计;二阶强度分析包括K函数等。

\par \textbf{样方法}:将研究区域划分为等大网格,统计每个网格内点的数量$\{g_i\}_{i=1}^G$,检验其是否符合Poisson分布。一种方法是利用Poisson分布均值等于方差的性质:
\begin{equation}
    I = \frac{\sum_{i=1}^G (g_i-\bar{g})^2}{(G-1)\bar{g}}
\end{equation}
CSR假设下近似有$(G-1)I \sim \chi_{G-1}^2$\footnote{当$G>6$且$\lambda\vert A_g\vert>1$时($A_g$为任一网格),可认为近似成立。}。$I$的显著高值代表聚集分布,显著低值代表规则分布。缺点是检验结果可能与网格大小有关。

\par \textbf{核密度估计}:可视为强度朴素估计(圆盘内计数与面积之比)的连续性扩展。核强度的估计式为
\begin{equation}
    \hat{\lambda}(\mathbf{x})=\sum_{i=1}^n \frac{1}{h^2} k\left(\frac{\mathbf{x}-\mathbf{x}_i}{h}\right)
\end{equation}
除以点计数即得核概率密度估计。其中带宽$h$可取局部值(随空间位置变化)。

\par \textbf{K函数}:若点过程满足二阶平稳、各向同性,二阶强度是距离的函数。定义K函数为
\begin{equation}
    K(d)=\frac{E_{\mathbf{x}\in X}(\vert X\cap U(\mathbf{x},d)\vert )}{\bar{\lambda}}
\end{equation}
其中$\bar{\lambda}$为平均强度. K函数的估计式为
\begin{equation}
    \hat{K}(d)=\frac{\vert R \vert}{n^2}\sum_{i=1}^n \vert X\cap U(\mathbf{x}_i,d)\vert
    \label{eq:hatK}
\end{equation}
若研究区域未包含完整的空间点过程,需要考虑边缘效应。当$d$相对研究区域较小时,可在式(\ref{eq:hatK})的求和号中忽略到边界距离小于$d$的点(分母相应变为$n(n-s)$,$s$为忽略的点个数);否则需要在和式的对应项除以调整因子以弥补边界带来的点数低估。CSR假设下有$K(d)=\pi d^2$;高于此值代表集聚,反之代表分散。可定义辅助函数与随机模式进行比较:
\begin{align}
    &L(d)=\sqrt{\frac{K(d)}{\pi}}-d\\
    &l(d)=\frac{1}{2}\ln\left(\frac{K(d)}{\pi}\right)-\ln(d)
\end{align}
显著性检验可采用自助法(估计样本分布)或Monte Carlo模拟(估计零假设下理论分布)。为检验两组空间点的分布模式是否有显著差异,可类似定义$l_{12}(d)=\ln K_1(d)-\ln K_2(d)$,利用自助法或Monte Carlo模拟(固定事件点位置,随机分配两类点的标签)检验显著性。

\par \textbf{交叉K函数}:度量一组空间点在另一组空间点周围聚集或排斥的趋势,定义为
\begin{align}
    &K_{12}(d)=\frac{E_{\mathbf{x}\in X_1}(\vert X_2\cap U(\mathbf{x},d)\vert )}{\bar{\lambda}_2}\\
    &\hat{K}_{12}(d)=\frac{\vert R \vert}{n_1n_2}\sum_{\mathbf{u}\in X_1}\sum_{\mathbf{v}\in X_2} I[\|\mathbf{u}-\mathbf{v}\|\le d]
\end{align}
若两组空间点来自同一个点过程,则交叉K函数为$\pi d^2$.

\section{空间回归分析}

\par 经典线性回归本质上是非空间的,用于空间数据时残差往往具有正空间自相关。误差项之间不独立使得OLS估计不再是有效的,统计推断也不能如常进行。建模非独立误差项的两种思路:(1)考虑误差项的空间自相关(空间滞后模型,Kriging回归);(2)建模误差项的空间趋势(半参数平滑法)。对(1)而言,这些方法给出的回归系数置信区间没有考虑空间权重矩阵(空间相关函数)的不确定性。

\par \textbf{一般框架}:模型记为
\begin{equation}
\mathbf{y}=X\boldsymbol{\beta}+\boldsymbol{\epsilon},\ \boldsymbol{\epsilon}\sim \mathcal{N}(0,C)
\end{equation}
$C$为任意协方差矩阵;误差不独立意味着$C$不是对角矩阵。若$C$已知,则有最大似然估计
\begin{equation}
    \hat{\boldsymbol{\beta}}=(X^{T}CX)^{-1}X^{\mathrm{T}} C\mathbf{y}
\label{eq:arbcmle}
\end{equation}
但$C$一般需要从样本数据中求出。

\par \textbf{空间自回归模型}(Spatial Autoregressive Model):
\begin{equation}
\mathbf{y}=\rho W \mathbf{y}+ X\boldsymbol{\beta}+\boldsymbol{\epsilon}
\end{equation}
其中$W$是行标准化的空间权重矩阵。上式可变形为
\begin{equation}
    \mathbf{y}=(I-\rho W)^{-1}X\boldsymbol{\beta}+(I-\rho W)^{-1}\boldsymbol{\epsilon}
\end{equation}
令$X'=(I-\rho W)^{-1}X$, $\boldsymbol{\epsilon}'=(I-\rho W)^{-1}\boldsymbol{\epsilon}$, 则变换后误差协方差矩阵为
\begin{equation}
    C = \sigma^2 (I-\rho W)^{-1}[(I-\rho W)^{-1})]^T
\end{equation}
采用最大似然法估计模型参数。

\par \textbf{空间误差模型}(Spatial Error Model)\footnote{原文称为“空间滑动平均模型”}:
\begin{equation}
\mathbf{y}=X\boldsymbol{\beta}+\mathbf{u},\ \mathbf{u}=\rho W\mathbf{u}+\boldsymbol{\epsilon}
\end{equation}
上式可变换为
\begin{equation}
(I-\rho W)\mathbf{y}=(I-\rho W)X\boldsymbol{\beta}+\boldsymbol{\epsilon}
\label{eq:semm}
\end{equation}
参数估计步骤为:(1)先视为经典线性回归,利用OLS估计得到初始系数$\boldsymbol{\beta}$; (2)计算残差,对残差估计自回归模型得到$\rho$; (3)利用式(\ref{eq:semm})和$\rho$值重新估计$\boldsymbol{\beta}$; (4)重复(2)-(3)直至收敛。

\par \textbf{Kriging回归}: 先视为经典线性回归,利用OLS估计得到初始系数$\boldsymbol{\beta}$,然后对残差拟合半变异函数$g(d)$,由此可得到误差的协方差矩阵$C$,并用式(\ref{eq:arbcmle})重新估计$\boldsymbol{\beta}$。事实上,
\begin{equation}
    g(d_{ij})=\frac{1}{2}\mathrm{E}(e_i-e_j)^2=\sigma^2-\operatorname{Cov}(e_i,e_j)=\sigma^2-c_{ij}
\end{equation}
其中$\sigma^2$是基台值(Sill),即半变异函数的渐近上界。

\par \textbf{半参数平滑法}:
\begin{equation}
    y_i=f(u_i,v_i)+\sum_j \beta_jx_{ij}+\epsilon_j
\end{equation}
其中$f$模拟误差的空间变化,由各样本点的残差根据距离衰减的核函数加权得到。矩阵形式为
\begin{equation}
\mathbf{y}=\mathbf{f}+X\boldsymbol{\beta}+\boldsymbol{\epsilon}, \ \mathbf{f}=W(\mathbf{y}-X\boldsymbol{\beta})
\end{equation}
对前一式用OLS估计,再代入后一式,得\footnote{若直接联立上述两式,则得到类似于空间误差模型的形式:$(I-W)\mathbf{y}=(I- W)X\boldsymbol{\beta}+\boldsymbol{\epsilon}$}
\begin{equation}
    \hat{\boldsymbol{\beta}}=(X^T(I-W)X)^{-1}X^T(I-W)\mathbf{y}
\end{equation}
其中带宽参数可采用广义交叉验证(Generalized Cross-Validation)选取。


\section{局部分析}

\par 全局分析是对研究区域的整体概括,可能隐藏空间变化。局部分析强调空间的差异性,其发展动力部分来源于对GIS与空间分析集成的兴趣(局部分析是空间显式的,其结果可用GIS可视化)。

\par \textbf{变量间关系表现出空间非平稳性的原因}:(1)随机采样的变化(\emph{也应包括测量误差等随机波动});并非关注重点,一般关注的是参数估计中的系统变化。(2)关系在空间上的本质差异;自然地理学中少见,但人的态度、偏好和行为是空间变化的,这一观点与后现代主义人类行为研究以位置和局部性为研究框架相符。(3)模型误设,如遗漏变量和函数形式误设;这一观点与实证学派相近,认为行为存在普遍规律,只是我们的模型不够完善。

\par \textbf{局部点模式分析}:点模式的全局统计量给出集聚性、分散性、随机性的总体描述,但无法探测局部异常,而后者在疾病研究中有重要应用。局部点模式分析的进展包括Openshaw提出的地理分析机(Geographical Analysis Machine)及其改进,该方法通过比较实际和期望情形下随机圆形窗口内点的个数识别局部异常。

\par \textbf{局部自相关分析}:可视化方法包括半方差函数云(semivariogram, 变量平方差对距离)和Moran散点图(邻域均值对当前值)。局部自相关指数包括Getis-Ord $G_i$指数和Moran $I_i$指数\footnote{Anselin指出,若要求全局与局部统计量有加和关系,$G_i$不属于局部空间自相关统计量。}。

\par \textbf{局部回归分析}:包括(1)空间自适应滤波(Spatial Adaptive Filtering),参数估计难以进行统计检验。(2)贝叶斯随机系数模型、多层次模型,没有关系空间依赖性的假设。(3)空间扩展法。(4)地理加权回归。

\par 多层次模型同时包含个体因素和地点因素(语境效应),避免仅考虑个体层面的微体谬误(Atomistic Fallacy)和仅考虑总体层面的生态谬误(Ecological Fallacy);需要定义离散的空间单元,且地理过程在边界处发生突变。

\par 空间扩展法的局限性在于需要预先设定方程的形式,且只能建模关系空间变化的整体趋势,可能掩盖局部变化。

\par 地理加权回归原理与局部加权回归(Locally Weighted Regression)、回归参数漂移分析(Drift Analysis of Regression Parameters, DARP)相似;后两种数据权重由属性相似度而非地理距离决定。

\par \textbf{局部空间交互模型}:交互起点为单元$i$的条件下,终点为单元$k$的概率
\begin{equation}
    p_{ik}=\frac{S_k^\alpha d_{ik}^{-\beta}}{\sum_j S_j^\alpha d_{ij}^{-\beta}}
\end{equation}
全局模型对参数$\alpha,\beta$作全局估计,也可对每个起点分别估计模型,得到参数的空间变化。

\section{空间数据统计推断}

\par 推断指基于可观测的相关信息对未观测的信息所作的陈述。推断方法大致可分为非形式推断(包括探索性数据分析、可视化、聚类分析等)和形式推断(包括经典统计推断、贝叶斯推断)。

\par 经典统计推断中的假设检验有时受到批评,如在$\theta$真实值很接近0时,若样本量足够大,$\theta=0$的零假设也会被拒绝。贝叶斯推断的争议主要在于先验分布的主观性,即可以根据预设结论调整先验;尽管在一定条件下当数据量足够大时,结果对先验选择不敏感。

\par 假设检验之间不相互独立给多重假设检验校正带来进一步的困难。可能的解决方案是考虑高维空间中的置信域(或贝叶斯统计中的多元后验分布)。

\section{空间交互模型}
\par 空间交互模型的发展可分为四个阶段:
\par \textbf{社会物理学阶段(1860-1970)}:Carey和Ravenstein指出城市间人口流动类似物体间的引力,因而出现了类比万有引力的重力模型:
\begin{equation}
    T_{ij}=k\frac{P_i^\alpha P_j^\gamma}{d_{ij}^\beta}
\end{equation}
其中$\alpha,\gamma$反映了空间流动与规模的待估关系;$\beta$建模了距离衰减作用对不同类型交互、不同地区的可能变化。上式的规模可扩展为一系列属性(方幂)的乘积,并用取对数最小二乘拟合参数。

\par \textbf{统计力学阶段(20世纪70年代)}:Wilson利用最大熵推导了空间交互模型族。设总交互量为$T$,熵定义为
\begin{equation}
    H=-\sum_{i,j}{\frac{T_{ij}}{T}\ln\frac{T_{ij}}{T}}
\end{equation}
无约束时交互量均匀分布于各边熵最大。Wilson引入一系列约束,不同约束组合分别推导出前述无约束、下述产出约束、吸引约束重力模型:
\begin{align}
    T_{ij}&=O_i P_j^{\gamma}d_{ij}^{-\beta}/\sum_k P_k^{\gamma}d_{ik}^{-\beta}\\
    T_{ij}&=D_j P_i^{\alpha}d_{ij}^{-\beta}/\sum_k P_k^{\alpha}d_{kj}^{-\beta}
\end{align}
以及双约束重力模型
\begin{align}
    T_{ij}&=A_iO_iB_jD_jd_{ij}^{-\beta}\\
    A_i &= \left(\sum_j B_jD_jd_{ij}^{-\beta}\right)^{-1}\\
    B_j &= \left(\sum_i A_iO_id_{ij}^{-\beta}\right)^{-1}
\end{align}
批评认为最大熵模型只是用统计力学的物理类比代替万有引力的物理类比,仍没有反映个体的空间决策过程;其部分约束条件也缺乏合理性。

\par \textbf{非空间信息处理阶段(20世纪80年代)}:MacFadden提出离散选择模型,根据目的地属性分配出发点$i$的流量。设目的地$j$的效用为$U_{ij}=V_{ij}+\mu_{ij}$,其中$V_{ij}$是可观测的部分,则选择目的地$j$的概率
\begin{equation}
    p_{ij}=P(U_{ij}>U_{ik}, \forall k \neq j)
\end{equation}
在一定条件下,有$p_{ij}=\exp{V_{ij}}/\sum_k \exp{V_{ik}}$。再令$V_{ij}\propto (\alpha \ln P_j)(-\beta \ln d_{ij})$,即得产出约束重力模型。这一模型限制了两个备选项概率之比不受其它选项影响;且新增目的地总会使其它选项概率降低(实际集聚效应可能导致其它选项概率升高)。

\par \textbf{空间信息处理阶段(20世纪90年代以来)}:Fotheringham提出了竞争目的地模型,基于人处理信息容量的有限性,假设人首先确定目的地的备选集,再从中选取目的地。具体而言,在模型中加入每个目的地进入备选集$M$的可能性度量,分配模型变为\begin{equation}
    p_{ij}=\frac{\exp{V_{ij}} \cdot l_i(j\in M)}{\sum_k \exp{V_{ik}} \cdot l_i(k\in M)}
\end{equation}
可能性度量的一种形式是利用可达性度量其它目的地的竞争效应:
\begin{equation}
    l_i(j\in M)= \left(\sum_{k\neq j}P_k^\alpha/d_{kj}^\beta\right)^\delta
\end{equation}
$\delta$一般为负,若为正,代表集聚效应。

\section{前景与展望}

\par 空间数据分析面临的挑战包括可变面状单元问题,分析结果依赖邻近关系(权重矩阵)的定义等。未来研究方向包括地理加权思想的扩展,贝叶斯推断的应用,时空相互作用(事件发生的时间、地点是否相互独立)等。

\par 感知或认知距离不太可能遵循欧氏几何规则。

