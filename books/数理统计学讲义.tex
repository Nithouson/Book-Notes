
\chapter{数理统计学讲义}

\par \Large \emph{陈家鼎\ 孙山泽\ 李东风\ 刘力平} 

\normalsize

\section{估计}

\subsection{总体参数的估计}

设随机变量$X$的概率分布函数为$F(x,\boldsymbol{\theta})$,给定$X$的样本$x_1,\dots,x_n$,要求估计参数$\boldsymbol{\theta}$。

\subsubsection{估计量的评价标准}

\par \textbf{相合与强相合}:当样本量$n\to +\infty$,若估计量依概率收敛到真值,称其有相合性;若估计量几乎必然(以概率1)收敛到真值,称其有强相合性。

\subsubsection{最大似然估计}

\par 由Ronald A. Fisher于1912年提出。设X的概率密度函数(离散随机变量为概率函数)为$f(x,\boldsymbol{\theta})$,则样本的联合密度函数为
\begin{equation}
L(x_1,\dots,x_n,\boldsymbol{\theta})=\prod_{i=1}^n f(x_i,\boldsymbol{\theta})
\end{equation}
若估计量$\hat{\boldsymbol{\theta}}=\boldsymbol{\varphi}(x_1,\dots,x_n)$满足
\begin{equation}
L(x_1,\dots,x_n,\hat{\boldsymbol{\theta}})=\sup_{\boldsymbol{\theta}}L(x_1,\dots,x_n,\boldsymbol{\theta})
\end{equation}
则称$\hat{\boldsymbol{\theta}}$是${\boldsymbol{\theta}}$的最大似然估计。上述条件等价于对数似然函数$\ln L(x_1,\dots,x_n,\boldsymbol{\theta})$的最大化。

\par 常见概率分布的最大似然估计:
\begin{itemize}
\item Bernoulli分布:$\hat{p}=\bar{x}=\sum_{i=1}^n x_i$.
\item 指数分布$\mathcal{E}(\lambda)$:$\hat{\lambda}=n/\sum_{i=1}^n x_i$.
\item 正态分布$\mathcal{N}(\mu,\sigma^2)$:$\hat{\mu}=\bar{x}$, $\hat{\sigma}^2=\frac{1}{n}\sum_{i=1}^n (x_i-\bar{x})^2$.
\item 均匀分布$\mathcal{U}(a,b)$:$\hat{a}=\min_i x_i$, $\hat{b}=\max_i x_i$.
\end{itemize}
上述最大似然估计量都是强相合估计,其中Bermoulli分布、指数分布、正态分布的情形可由强大数律导出。


\subsubsection{矩估计}

\par 由Karl Pearson于1894年提出。设$\boldsymbol{\theta}=(\theta_1,\dots,\theta_m)$,$X$的1至$m$阶原点矩存在,记为$V_k=EX^k=g_k(\theta_1,\dots,\theta_m)$, $k=1,\dots,m$. 若能反解得到$\theta_j=f_j(V_1,\dots,V_m)$,$j=1,\dots,m$,则代入样本矩$\Tilde{V}_k=\frac{1}{n}\sum_{i=1}^n x_i^k$,即得到参数的矩估计。有时也用中心矩$E(X-EX)^k$做矩估计。由强大数律,样本矩几乎必然收敛到总体矩;故当所有$f_j$为连续函数时,矩估计量是强相合的。

\par 正态分布的矩估计与最大似然估计一致,但这并非普遍情形。

\section{假设检验}

\subsection{补充}

\subsubsection{Mann-Whitney U检验}

\subsubsection{Kolmogorov–Smirnov检验}